\chapter{Fourier-analytic Barron spaces}
\label{sec:fourier}

In this chapter, we study the approximation property. As shown in
\cite{barronUniversalApproximationBounds1993}, 
In Chapter \TODO, the question of \textit{density} has been answered for 2NN in
which functions on any compact domain in $\R^d$ are dense in 2NN w.r.t. the
supremum norm . Barron presented an different
approach where given a dictionary $\mathbb{D}$, one tries to find the class of
functions that are \textit{well approximated} by such dictionary.

This chapter is organized as follows. In section \ref{sec:spectral_condition},
the class of functions that satisfies certain smoothness constrain is
introduced. Section \ref{sec:construction_of_fouier} defines and characterizes
the function spaces constructed based on the previous smoothness restriction. In
section \ref{sec:approximation_rate_fouier}, an approximation rate of
$\bigO(n^{-\frac{1}{2}})$ is proven w.r.t. the supremum norm where $n$ is the
number of nodes. Section \ref{sec:fourier_variation_space} shows the dictionary
corresponding to the smoothness condition. Finally, we give some high order
error rate for ReLU\textsuperscript{k} networks

To apply the argument of nonlinear approximation with $n$-term dictionary to
2NN, we consider the dictionary
\begin{equation}
    \mathbb{D}_{\sigma} := \{
        \sigma(b\tr x + c), \quad b\in\R^d, c\in\R
    \}
\end{equation}
for any $x \in U$ where $U$ is a nonempty bounded domain in $\R^d$, and
$\sigma:\R\to\R$ is a bounded activation function.

Let $\Sigma_n(\mathbb{D}_{\sigma})$ be the class of function of $n$-width 2NN
with activation function $\sigma$
\begin{equation}
    \Sigma_n(\mathbb{D}_{\sigma}) = \Sigma_n^{\sigma}
    := \Bigg\{
        \sum_{j=1}^n a_j d_j, \quad a_j\in\R, d_j\in \mathbb{D}_{\sigma}
    \Bigg\}.
\end{equation}

If a function $f$ is in the closed convex hull of $\mathbb{D}_{\sigma}$
\begin{equation}
    \closure{\conv(D_{\sigma})} :=
    \closure{
        \Bigg\{ 
            \sum_{j=1}^n a_j d_j: 
            n\in\Nat, a_j\in\R, d_j\in\mathbb{D}_{\sigma}, 
            \sum_{j=1}^n \norm{a_j}_1 \leq 1
        \Bigg\}
    }.
\end{equation}
Applying the Maurey's Theorem (\TODO or should I say the sampling argument?)
yields a approximation rate
\begin{equation}
    \inf_{f_n\in\Sigma_n^{\sigma}} \norm{f - f_n} 
    \lesssim n^{-\frac{1}{2}}
\end{equation}
where $f$ is any function in the closed convex hull of $\mathbb{D}_{\sigma}$
and $2 \leq q < \infty$.

This chapter will show that functions that satisfies certain smoothness
conditions are in $\closure{\conv(D_{\sigma})}$ where $\sigma$ is any bounded
sigmoidal activation functions.


\section{Barron class and spectral condition}
\label{sec:spectral_condition}

In this section, we would identify the class of functions originally proposed by
\cite{barronUniversalApproximationBounds1993}. We define the \textit{spectral
condition} on which the smoothness constrain is imposed. A modified spectral
condition proposed by \cite{siegelCharacterizationVariationSpaces2022} is also
introduced.

\begin{definition}[Barron class]
    \label{def:barron_class}
    Let $U$ be a nonempty bounded domain in $\mathbb{R}^d$. A function $f: U \to
    \mathbb{R}$ is said to be in \textit{Barron class} with a constant $C > 0$,
    if there is a $x_0$ in $U$, $c \in [-C, C]$, and a measurable function $f:
    \mathbb{R}^d \to \mathbb{C}$ satisfying:
    \begin{align}
        & \int_{\mathbb{R}^d} \abs{\omega}_{U, x_0} 
        \cdot \abs{\fourier{f}(\omega)} \,d\omega < C 
        \label{eq:another_spectral_seminorm_condition} \\
        & f(x) = c + \int_{\mathbb{R}^d} (
            e^{i\omega\tr x} - e^{i\omega\tr x_0}
        ) \cdot \fourier{f}(\omega)\,d\omega
    \end{align}

    where $\abs{\omega}_{U, x_0} := \sup_{x\in U}\norm{\spr{\omega}{x - x_0}}$
    and it is denoted by $\abs{\omega}_U$ when $x_0 = 0$ for simplicity. We
    refer the class of all functions satisfying the above condition as 
    $\Gamma_C(U, x_0)$\footnote{
        $\Gamma_C(U, x_0)$ is the Fourier-analytic Barron space
        $\bspace{\mcal{F}, 1}(U)$ discussed before. The integral condition in
        \eqref{eq:another_spectral_seminorm_condition} is the integral condition
        $v_{f,1} < \infty$ defined in \eqref{eq:spectral_condition}.
    }.
\end{definition}

\begin{remark}
    In the above definition, the domain $U$ bounded using $\abs{\omega}_{U}$ is
    interpreted as bounding the trigonometric component $e^{i\omega\tr x}$ where
    $\omega$ is the frequency in the Fourier representation of the functions
    restricted. It it easier to see that the constant $C$
    \begin{equation}
        C \leq r \int_{\R^d} \abs{\omega} \abs{\fourier{f}(\omega)}\,d\omega.
    \end{equation}
    if $U$ is in a ball of radius $r$ since $\abs{\omega}_{U} \leq r \cdot
    \abs{\omega}$ by the Cauchy-Schwarz inequality.

    More generally, if $U$ is a $l_p$ ball of radius $r$, we can rewrite the
    condition as
    \begin{equation}
        r \int_{\R^d} \norm{\omega}_p \abs{\fourier{f}(\omega)}\,d\omega 
        < \infty, \quad 1 \leq p \leq \infty.
    \end{equation}

\end{remark}


% Here it is the L1 norm of \omega

\begin{definition}[Spectral condition]
    \label{def:spectral_condition}
    Let $U$ be a nonempty bounded domain in $\R^d$.
    Suppose that a function $f: U \to \R$ admits a Fourier representation
    \begin{equation}
        f(x) = \int_{\R^d} e^{i\omega\tr x} \fourier{f}(\omega) \,d\omega
    \end{equation}
    where $\fourier{f}: \R^d \to \mathbb{C}$ is the Fourier transform of $f$.

    For any $s \in \Nat$, the spectral condition of $f$ is defined as
    \begin{equation}
        \label{eq:spectral_condition}
        v_{f,s} 
            = \int_{\R^d} \abs{\omega}^s \abs{\fourier{f}(\omega)}\,d\omega.
    \end{equation}
\end{definition}



\begin{definition}[Modified spectral condition]
    \label{def:spectral_condition_mod}
    Under the setup of Definition \ref{def:spectral_condition}, we define a
    modified spectral condition
    \begin{equation}
        \label{eq:spectral_condition_mod}
        v'_{f,s} 
            = \int_{\R^d} (1 + \abs{\omega})^s \abs{\fourier{f}(\omega)}
            \,d\omega.
    \end{equation}
\end{definition}



\section{Construction of Fourier-analytic Barron spaces}
\label{sec:construction_of_fouier}

Firstly, we define the spectral condition and the seminorm and norm\footnote{
    Often papers include ``Barron'' as readers can deduce from the context. 
    In the following chapters, we would like to call them \textit{spectral 
    semi/norm} as another definition of norm in infinite-width Barron spaces is 
    named \textit{Barron norm}.
} by which the smoothness of a function
is controlled. 

\begin{definition}[Spectral seminorm]
    \label{def:spectral_seminorm}
    Let $U$ be a nonempty bounded domain in $\R^d$. Suppose that a function $f:
    U \to \R$ admits a Fourier representation
    \begin{equation}
        f(x) = \int_{\R^d} e^{i\omega\tr x} \fourier{f}(\omega) \,d\omega
    \end{equation}
    where $\fourier{f}: \R^d \to \mathbb{C}$ is the Fourier transform of $f$.

    For any $s \in \Nat$, the spectral seminorm of $f$ is defined as
    \begin{equation}
        \specseminorm{f}{s} = \inf_{f_{e\mid U} = f} v_{f,s}
    \end{equation}
    where the infimum is taken over all extensions $f_e$ of $f$ in $\lp1(U)$.
\end{definition}


The notion of spectral norm was first introduced in
~\cite{siegelApproximationRatesNeural2021} since it is more convenient compared
to the seminorm.

\begin{definition}[Spectral norm]
    \label{def:spectral_norm}
    Under the setup of Definition \ref{def:spectral_seminorm}, we define a
    spectra norm
    \begin{equation}
        \specnorm{f}{s} = \inf_{f_{e\mid U} = f} v'_{f,s}.
    \end{equation}
\end{definition}

\begin{definition}[Fourier-analytic Barron spaces]
    \label{def:fourier_space}
    Let $U$ be a nonempty bounded domain in $\R^d$. The Fourier-analytic Barron
    spaces are
    \begin{equation}
        \bspace{\mcal{F},s}(U) := \Big\{
            f: U \to \R: v'_{f,s} < \infty  \textnormal{ and }
            \forall x\in U, 
                f(x) = \int_{\R^d} e^{i\omega\tr x} \fourier{f}(\omega)\,d\omega
        \Big\}
    \end{equation}
    equipped with a norm $\specnorm{f}{s}$ for all $s \in \Nat$.
\end{definition}

\TONOTE{What is the extensions}

\TONOTE{Why is 1/2 allowed? in \cite{siegelCharacterizationVariationSpaces2022}}

$s \in \Nat$ is referred as the smoothness index. It is easy to conjecture
that as $s$ increases, functions in $\bspace{\mcal{F}, s}$ are becoming
``smoother''. Therefore the \textit{size} of the function spaces of higher
smoothness index $s$ is expected to ``shrink'' which might suggests better
approximation error rate. In the coming sections, we will provide an accurate
statement of the improved approximation error rates with a depiction of the
intricacies inherent within and between these spaces.

\TODO

In general, the activation function associated with the infinite-width Barron
spaces is ReLU and we will explicitly state when other functions (e.g. squared
ReLU, ReLU\textsuperscript{k}, Heaviside) are used.

The notation of Barron spaces is not in consensus within the community and
different terms have been given to describe the same model classes or spaces.
For the function spaces in which functions have finite Fourier moments,
~\cite{xuFiniteNeuronMethod2020} call this model classes \textit{Barron spectral
spaces} while \cite{carageaNeuralNetworkApproximation2022} refers them as
\textit{Fourier-analytic Barron space}.  For function spaces in which functions
admit an integral representation \eqref{eq:barron_represent} with a ReLU
activation function, \cite{eBarronSpaceFlowinduced2021} refer them simply as
\textit{Barron spaces} of different orders $p \in \{1, 2, 3, \cdots, \infty\}$.
The term \textit{Barron space} was coined by
\cite{ePrioriEstimatesPopulation2019} to honor Prof. Andrew Barron's
contribution in the understanding of neural nets. In some literature
including~\cite{carageaNeuralNetworkApproximation2022}, these spaces are named
\textit{infinite-width Barron spaces} associated with different activation
functions and the term \textit{classical Barron space} is reserved for those
associated with Heaviside function\footnote{Also called step function or unit
step function}.

To avoid confusion and in the meantime emphasize Prof. Andrew Barron's
contribution, we will use two definitions:
\begin{itemize}
    \item Fourier-analytic Barron spaces
    \item infinite-width Barron spaces\footnote{
        We limit the model classes to those associated with ReLU only. In the 
        case of Heaviside function, we denote the space still by the term 
        \hyperref[def:heaviside_space]{\textit{classical Barron space}}.
    }
\end{itemize}


\section{Approximation rate in Fourier-analytic Barron spaces}
\label{sec:approximation_rate_fouier}

It is shown by \cite{barronUniversalApproximationBounds1993} that functions of
$d$-variables with finite Fourier moments can be approximated with the
superpositions of sigmoidal functions at a rate independent of the
dimensionality $d$. The result is extended to ReLU activation function. In other
words, any functions in that class can be approximated with a 2NN at an error
rate of $\bigO(n^{-1} \cdot C)$ where $n$ is the number of nodes in the the
single hidden layer and $C$ is a constant dependent \textit{only} the smoothness
of the target function. Despite that the convergence rate itself is independent
of dimension (i.e. the dimensionality $d$ of input vector $x\in\R^d$), the
constant $C$ could be dimension-dependent as the Fourier transform is used here.

This section is based on \cite{barronUniversalApproximationBounds1993}.

\begin{theorem}
    % \cite[Theorem~1]{barronUniversalApproximationBounds1993}
    \label{thm:barron_1993_1}
    Let $U$ be a nonempty bounded domain in $\R^d$, $x_0 \in U$, and $C > 0$ a
    constant. For every function in the Barron class $\Gamma_C(U, x_0)$, every
    sigmoidal function $\phi$, every probability measure $\mu$, and every $n \in
    \Nat$, there exists a linear combination of sigmoidal function $f_n(x) =
    \sum_{j=1}^n a_j \sigma(b_j\tr x+c_j), a_j, c_j \in \R, b_j \in \R^d$ such
    that
    \begin{equation}
        \norm{f - f_n}_{\lp{2}(U)}\leq n^{-\frac{1}{2}} \cdot 2C
    \end{equation}
\end{theorem}

\begin{proof}
    The main idea behind the the proof is to show functions with finite Fourier
    moment are in the closure of the convex hull of the set of half planes.

    \textbf{Step 0} (\textit{Fix $x_0$ to $0$}): Let $x_0, x_1$ be two
    arbitrarily selected points in $U$, and $f \in \Gamma_C(U, x_0)$.  For any
    $\omega \in \mathbb{R}^d$, given $x_0, x_1$, we have
    \begin{equation}
        \abs{\omega}_{U, x_0} 
            = \sup_{x\in U}\norm{\spr{\omega}{x-x_0}} 
            \leq \sup_{x\in U}\norm{\spr{\omega}{x-x_1}} + \norm{\spr{\omega}{x_0-x_1}} 
            \leq 2\abs{\omega}_{U, x_1}
    \end{equation}

    Therefore, we have $\int_{\mathbb{R}^d} \abs{\omega}_{U, x_1}
    \abs{\fourier{f}(\omega)}\,d\omega \leq 2C$. If we have $\tilde{c} = c +
    \int_{\mathbb{R}^d} (e^{\omega\tr x_0} - e^{\omega\tr x_1}) d\omega$, then
    $f(x) = \tilde{c} + \int_{\mathbb{R}^d} (e^{i\omega\tr x} - e^{i\omega\tr
    x_1})$ with $\tilde{c} \leq 2C$.

    This shows that changing $x_0$ would only affect the constant in the RHS of
    Theorem \ref{thm:barron_1993_1} by a factor of at most two, i.e.
    $\Gamma_C(U, x_0) \subset \Gamma_{2C}(U, x_1)$. Therefore, we continue the
    proof assuming $x_0 = 0$.

    \textbf{Step 1} (\textit{Represent $f$ via Inverse Fourier Transform}): With
    the polar decomposition, we have $\fourier{f}(\omega) = e^{i\theta(\omega)}
    \cdot \abs{\fourier{f}(\omega)}$ where $\theta(\omega) \in \mathbb{R}$
    denote the magnitude decomposition. From the assumption, and the fact that
    $f$ is real-valued ($f: U \to \mathbb{R}$), the real-valued part of $f(x) -
    f(0)$ can be written as:
    \begin{align}
        \label{eq:barron_fouier_int}
        f(x) - f(0)
        & = \Re \int (e^{i\omega\tr x} - e^{i\omega\tr 0}) e^{i\theta(\omega)} \cdot 
        \abs{\fourier{f}(\omega)} \,d\omega \\
        & = \int_{\Omega}\Big(\cos(\omega\tr x + \theta(\omega)) - \cos(\theta(\omega))\Big)
        \abs{\fourier{f}(\omega)} \,d\omega \\
        & = \int_{\Omega} \frac{C_{f,U}}{\abs{\omega}_{U, 0}}\Big(\cos(\omega\tr x + \theta
        (\omega)) - \cos(\theta(\omega))\Big)\,d\mu_g \\
        & = \int_{\Omega} g(x, \omega)\,d\mu_g.
    \end{align}
    where we denote $\int_{\mathbb{R}^d} \abs{\omega}_{U} \cdot
    \abs{\fourier{f}(\omega)} d\omega \leq C$ by $C_{f, U}$.

    $\mu_g$ is a probability distribution $d\mu_g =
    \abs{\omega}_{U}/C_{f,U}\abs{\fourier{f}(\omega)} d\omega$, the integral is
    evaluated on $\Omega = \{\omega \in \R^d: \omega \not = 0\}$ and
    \begin{equation}
        g(x, \omega) = \frac{C_{f,U}}{\abs{\omega}_{U}}
        \Big(
            \cos(\omega\tr x + \theta(\omega)) - \cos(\theta(\omega))
        \Big).
    \end{equation}

    \textbf{Step 2} (\textit{$f(x) - f(0)$ is in the closure of the convex hull
    of $G_{\cos}$}): The integral form in (\ref{eq:barron_fouier_int}) shows that
    $f(x) - f(0)$ can be represented as an infinite convex combination of
    functions in the class

    \begin{equation}
        G_{\cos} = \Bigg\{
            \frac{\abs{\gamma}}{\abs{\omega}_{U}}
            \Big(
                \cos(\omega\tr x + b) - \cos(b)
            \Big):
                \omega \not= 0, \abs{\gamma} \leq C, b \in \mathbb{R} 
        \Bigg\}
    \end{equation}

    Suppose we have drawn $n$ samples ($\{\omega_i, i = 1,\dots, n\}$) from
    $\mu_g$, the expected norm in $\lp{2}(U, \mu_g)$ converges to zero as $n \to
    \infty$ by $\lp{2}$ law of large numbers. Therefore, there exist a convex
    combination of elements in $G_{\cos}$ that converges to $f(x) - f(0)$ in
    $\lp{2}$.


    \textbf{Step 3} (\textit{$G_{\cos}$ is in the closure of the convex hull of
    $G_{\textnormal{step}}$}): It is sufficient to check $g(z), z = \alpha x,
    \alpha = \omega/\abs{\omega}_{U}$ on $[-1, 1]$ for some $\omega \not= 0$. As
    $g(z)$ is a uniformly continuous sinusoidal function on $[-1, 1]$, it can be
    uniformly approximately by piecewise constant step function.

    Restricting $g(z)$ on $[0, 1]$, for a partition ${0 \leq p_1 \leq p_2 \leq
    \cdots \leq p_k = 1}$, define

    \begin{align}
        g_{k,+}(z) = \sum_{i=1}^{k-1} \Big(g(p_i) - g(p_{i-1})\Big) \cdot
        \indicator{\{z\geq p_i\}}(z)
    \end{align}

    Similarly, we can construct $g_{k,-}(z) = \sum_{i=1}^{k-1} (g(-p_i) -
    g(-p_{i-1})) \cdot \indicator{\{z\leq -p_i\}}(z)$, resulting in a sequence
    of piecewise step function on $[-1, 1]$ uniformly close to $g(z)$. We have
    $g(z) = g_{k,+}+g_{k,-}$, a linear combination of step function (or
    heaviside function) and the sum of the coefficients is bounded by 2C (The
    sum of coefficients of $g_{k,+}$ is bounded by $C$ as a result of the
    derivative of $g$ bounded by $C$, so does $g_{k,-}$ and hence $2C$).

    We can see that functions $g(z)$ are in the closure of the convex hull of
    the step functions (by Lemma 1 in
    \cite{barronUniversalApproximationBounds1993})

    By substituting $z = \frac{\omega}{\abs{\omega}_{U}} x$, we have $G_{\cos}
    \subset G_{\textnormal{step}}$,
    \begin{equation}
        G_{\textnormal{step}} = \Bigg\{
            \gamma\indicator{\{\alpha x-t\}}(x):
            \abs{\gamma} \leq 2C,
            \abs{t} \leq 1,
            \abs{\alpha}_{U} = 1
        \Bigg\}.
    \end{equation}

    \textbf{Step 4} (\textit{Closure of $G_{\phi}$}): There exists a sequence of
    sigmoidal functions $\phi(\abs{c}(\alpha x - t))$, as $\abs{c} \to \infty$,
    they converge to step functions pointwise (except at points where $\alpha x
    - t = 0$). If we introduce a measure $\mu$ that has zero measure at those
    points, previous statement on $G_{\cos} \subset G^{\mu}_{\textnormal{step}}$
    still holds on $\{\abs{t} \leq 1: \alpha x - t \not=0\}$ given a particular
    $\alpha$. We subsequently have convergence in $L_2(U, \mu)$ by the Dominated
    Convergence Theorem, which implies that $G^{\mu}_{\textnormal{step}} \subset
    G_{\phi}$.

    Finally, we arrive at the following relationship since the closure of a
    convex set is also convex \eqref{def:closed_convex_hull}
    \begin{equation*}
        \Gamma_{U, x_0} \subset \closure{G_{\cos}} 
        \subset \closure{G_{\textnormal{step}}} \subset \closure{G_{\phi}}.
    \end{equation*}

    % \cite[\textit{Lemma~1}]{barronUniversalApproximationBounds1993}: If $f$ is


    % Lemma 1 in \cite{barronUniversalApproximationBounds1993} showed that
    % function in a closure of the convex hull of a set in a Hilbert space can
    % be approximated with a sequence of functions from such closure and the
    % norm between the function and the sequences $\{f_i, i = 1, \dots, N\}$ are
    % bounded in a magnitude of $\bigO(N^{-1})$.

    It has been shown above that function $f(x) - f(0)$ is in the closure of
    the convex hull of $G_{\phi}$ where $\norm{g} \leq (2C)^2$ for every $g \in
    G_{\phi}$. Hence the $L_2$ norm of the approximation error is bounded for
    any choice of $C' > (2C)^2 - \norm{f(x) - f(0)}^2$ by Corollary
    \ref{cor:maurey}.
    % Suppose we restrict $t$ t

    % Suppose we now restrict $t$ to the continuity point induced by measure
    % $\mu$ in We can check that the functions in $G_{\textnormal{step}}$ are in the closure
    % of the convex hull of 

    % \textbf{Step 3} (\textit{Putting it together}): We can further show
    % $G_{\cos}$ are in the class of sigmoidal functions. 

    % Theorem 2 in \cite{barronUniversalApproximationBounds1993}, we have that

\end{proof}


\section{Connection with variation spaces}
\label{sec:fourier_variation_space}

What is separable? Is it necessary?

Let $(\mcal{H}, \norm{\cdot}_{\mcal{H}})$ be a separable Hilbert space.

If $G$ is a subset of $\mcal{H}$ and $c \in \R$, then we define the set
\begin{equation}
    cG = \{cg: g \in G\}
\end{equation}


\begin{definition}[Varitaion norm]
    The variation norm of $\normVar{f}{\mathbb{D}}$ of a subset $\mathbb{D}$ of
    a linear space $X$ is defined for all $f \in X$ as
    \begin{equation}
        \normVar{f}{\mathbb{D}} := \inf \{
            c > 0: f/c \in \closure{\conv(\mathbb{D} \bigcup - \mathbb{D})}
        \}.
    \end{equation}

    This is the Minkowski functional of the closed symmetric convex hull of $\mathbb{D}$
    \begin{equation}
        \closure{\conv(\mathbb{D} \bigcup - \mathbb{D})} := \Bigg\{ 
            \sum_{j=1}^n a_j d_j: n \in \Nat, d_j \in \mathbb{D}, 
            \sum_{j=1}^n \norm{a_j}_1 \leq 1
        \Bigg\}.
    \end{equation}
\end{definition}

\begin{definition}[Varitaion space]
    The variation space $\spaceVar{\mathbb{D}}$ is given by
    \begin{equation}
        \spaceVar{\mathbb{D}} := \{ 
            f \in \mathcal{H}: \normVar{f}{\mathbb{D}} < \infty
        \}.
    \end{equation}
\end{definition}

By the definitions, the following elementary properties of the variation space
hold.
\begin{proposition}
    \label{prop:spaceVar_properties}
    Let $\mathbb{D}$ be a uniformly bounded subset of a Hilbert space $\mcal{H}$
    \begin{equation}
        \sup_{d\in\mathbb{D}} \norm{d}_{\mcal{H}} < \infty.
    \end{equation}
    \begin{enumerate}
        \item $\closure{\conv(\mathbb{D} \bigcup - \mathbb{D})} = \{
            f\in\mcal{H}: \normVar{f}{\mathbb{D}} \leq 1
        \}$
        \item $\norm{f}_{\mcal{H}} \leq \normVar{f}{\mathbb{D}} \cdot
        \sup_{d\in\mathbb{D}} \norm{d}_{\mcal{H}}$
        \item $\spaceVar{\mathbb{D}}$ is a Banach space equipped with norm
        $\normVar{f}{\mathbb{D}}$
    \end{enumerate}
\end{proposition}

\begin{proof}
    (1) and (2) is clear from the previous definitions. In order for
    $\spaceVar{\mathbb{D}}$ to be a Banach space, we need to show
    $\spaceVar{\mathbb{D}}$ is complete with norm $\normVar{\cdot}{\mathbb{D}}$.

    Let $\{f_n\}$ be a Cauchy sequence w.r.t. $\normVar{\cdot}{\mathbb{D}}$. By
    (2), this automatically implies that $f_n \to f$ in $\mcal{H}$ so the
    sequence is Cauchy w.r.t. $\norm{\cdot}_{\mcal{H}}$.
\end{proof}



\begin{proposition}
    \TONOTE{Is H separable?}

    Let $(\mcal{H}, \norm{\cdot}_{\mcal{H}})$ be a Hilbert space and $f$ be a
    function in $\mcal{H}$. Suppose that a sequence $\{f_n\}$ in $\mcal{H}$
    where $f_n \in \Sigma_{n,M}(\mathbb{D})$ \TODO and $f_n \to f$ in $\mcal{H}$
    for some fixed $M < \infty$. Then $f$ is in the variation space
    $\spaceVar{\mathbb{D}}$ and its variation norm is bounded by $M$
    \begin{equation}
        f \in \spaceVar{\mathbb{D}}, \quad \normVar{f}{\mathbb{D}}\leq M.
    \end{equation}
\end{proposition}

\begin{proof}
    Without loss of generality, we only need to prove for $M = 1$.
\end{proof}


Here we would like to connect $\mathbb{D}$ with the integral representation. An integral representation of a function $f$ using dictionary $\mathbb{D}$ is
\begin{equation}
    f = \int_{\mathbb{D}} \inclusionMap{\mathbb{D}}{\mcal{H}} \,d\mu.
\end{equation}

Here, $\mu$ is a signed Borel measure on $\mathbb{D}$ with finite variation on $\mathbb{D}$:
\begin{equation}
    \label{eq:measure_set_variation}
    \norm{\mu}_{\mathbb{D}} := \sup_{\substack{
            g: \:\mathbb{D} \to [-1, 1] \\ g \textnormal{ is measurable}
        }
    } \int_{\mathbb{D}} g\,d\mu < \infty.
\end{equation}


Here I need to add some argument on why the map is integrable

\begin{definition}[Inclusion map]
    \label{def:inclusionMap}
    Let $A \subseteq B$. The injection $\inclusionMap{A}{B}: A\to B$ is an
    inclusion map if
    \begin{equation}
        \inclusionMap{A}{B}(a) = a, \quad \forall a \in A.
    \end{equation}
\end{definition}

\begin{lemma}
    \label{lemma:compact_set_integral_representation}
    Let $\mathbb{D}$ be a compact subset on $\mcal{H}$. Then $f \in
    \spaceVar{\mathbb{D}}$ if and only if there exists a Borel measure $\mu$ on
    $\mathbb{D}$ such that
    \begin{equation}
        f = \int_{\mathbb{D}} \inclusionMap{\mathbb{D}}{\mcal{H}} \,d\mu
    \end{equation}
    where $\inclusionMap{\mathbb{D}}{\mcal{H}}$ is an inclusion map from
    $\mathbb{D}$ to $\mcal{H}$. Furthermore, the variation norm of $f$ is given
    by the infimum of measure's variation over all the signed Borel measure on
    $\mathbb{D}$:
    \begin{equation}
        \normVar{f}{\mathbb{D}} = \inf_{\mu}\Big\{
            \norm{\mu}_{\mathbb{D}}: 
            f = \int_{\mathbb{D}} \inclusionMap{\mathbb{D}}{\mcal{H}} \,d\mu
        \Big\}.
    \end{equation}
\end{lemma}

\begin{remark}
    the compactness is necessary as $\bspace{\mcal{H}}$ is not included by the
\end{remark}

\begin{definition}[Spectral dictionary]
    The spectral dictionary of order $s \in \Nat^0$ is given by
    \begin{equation}
        \mathbb{F}_s := \Bigg\{ 
            (1 + \norm{\omega})^{-s} \cdot e^{2\pi i\omega\tr x}: 
            \omega \in \R^d
        \Bigg\}
    \end{equation}
\end{definition}

The variation space of $\mathbb{F}_s$ can be characterized in terms of the
Fourier transform.

The variation space $\spaceVar{\mathbb{F}_s}$ corresponding to the spetral
dictionary $\mathbb{F}_s$.

I suppose that the domain $U \subseteq \R^d$ is required to be compact?
It seems like $U$ is only bounded.

\begin{theorem}[Equal norm]
The Fourier-analytic Barron space is equivalent to the variation space
constructed with $\mathbb{F}_s$
\begin{equation}
    \bspace{\mathcal{F}, s} = \spaceVar{\mathbb{F}_s}.
\end{equation}

Furthermore, the spectral norm is equivalent to the variation norm:
\begin{equation}
    \normVar{f}{\mathbb{F}_s} =
    \inf_{f_{e\mid U} = f} \int_{\R^d} 
    (1+\abs{\omega})^s \abs{\fourier{f}} \,d\omega.
\end{equation}
\end{theorem}

\begin{proof}
    
\end{proof}


\section{Improved rate via Heaviside function}
\label{sec:improved_heaviside}

In the following sections, we present an exposition on the improved
approximation error rates either by compactness condition of the dictionary set
or stricter smoothness condition. The curx of the notion is to derive the
entropy number corresponding to the convex hull of the dictionary $\mathbb{D}$.
Maurey's Theorem \ref{thm:maurey} provides a bound for \textit{any} bounded
dictionaries in a Hilbert space. The improvement was first done in
\cite{makovozRandomApproximantsNeural1996} by the compactness of the dictionary
of Heaviside networks. Subsequently,
\cite{klusowskiApproximationCombinationsReLU2018} showed ReLU and squared ReLU
networks where the compactness of the dictionary is guaranteed by controlling
the $\lp{1}$ and $\lp{\infty}$ norm of parameters. In the latter, the
\textit{inner parameters} $b\in\R^d$ in \eqref{eq:improved_fourier_dict} are
suitably constrained to ensure compactness. The works of
\cite{maUniformApproximationRates2022, siegelSharpBoundsApproximation2022,
klusowskiApproximationCombinationsReLU2018a} provide the basis for our
discussion, unless explicitly cited otherwise.

\cite{makovozRandomApproximantsNeural1996} showed that the error can be improved
to $\bigO(n^{\frac{1}{2} - \frac{1}{p \cdot d}})$ in $\lp{p}(\Omega)$, $p <
\infty$.

\begin{theorem}
    % \cite[Theorem 3]{makovozRandomApproximantsNeural1996}
    \label{thm:improve_barron}
    Let $U$ be bounded set on $\R^d$. Let $f: U \to \R$ be a function and $V$ be
    the closure of all functions $f: U \to \R$ of the following form
    \begin{equation}
        V = \closure{
            \{f(x) = \sum_{j=1}^n a_j H(b_j\tr x + c_j) \quad 
            \sum a_j \leq 1, \abs{b_j} = 1, c_j \in \R \}
            }
    \end{equation}
    where $H$ is the Heaviside function.

    Then there exists a finite linear combination $f_n$ 
    \begin{equation}
        f_n = \sum_{j=1}^n a_j \sigma(b_j\tr x + c_j), b_j \in R^d, a_j, c_j \in \R
    \end{equation}
    where $\sigma(x)$ is a sigmoidal function from $\R$ to $\R$.

    For any $f \in V$ and any $n \in \Nat$ such that 
    \begin{equation}
        \norm{f- f_n}_{\lp{p}(U)} \leq C n^{-\frac{1}{2} - \frac{1}{p \cdot d}}
        \quad 1 \leq p < \infty
    \end{equation}
    where $C$ is a constant dependent only $U$, $p$ and $s(\cdot)$.
\end{theorem}

% In some literature, V is represented as the dictionary

Note that this theorem does not cover the case $p = \infty$. However,
\cite{barronUniversalApproximationBounds1993} has already showed that $f\in V$ are
dense w.r.t. supremum norm ($\norm{f - f_n}_{\infty} = \bigO(n^{-1/2})$), which
implies $\norm{f-f_n}_{q}$ for all $\lp{q}, q<\infty$.

As the dimensionality $d$ increases, this rate approaches to the original rate
and therefore this theorem is insignificant for high dimensionality. It is
sufficient to prove the case $s$ is sigmoidal. It is easy to check $s(\lambda t)
\to s(t)$ as $\lambda \to \infty$ and $s(\lambda t) \to s{t}$ as $\lambda \to
-\infty$. On a closed interval $[-u, u]$ on $\R$, note that the difference $s(t)
- \sigma(\lambda t)$ is still bounded everywhere. Therefore, we can see that
distance between $s$ and $\sigma$ can be made arbitrarily small on the space
$\lp{d}(\R)$ with a sufficiently large $b$.

\begin{proof}
    We denote the set of sigmoidal functions
    \begin{equation}
        A = \{\sigma(b,c): \sigma(b,c) = \sigma(b\tr x + c), \quad 
        b\in\R^d, c\in\R\}
    \end{equation}

    As we already knew from \cite{barronUniversalApproximationBounds1993} that
    $V$ is the closure of the convex, symmetric hull of $A$ in $\lp{p}(U)$
    \begin{equation}
        V = \closure{\conv(A \bigcup - A)}.
    \end{equation}
    
    Now it remains to finding an estimate for the entropy of $A$. Without loss
    of generality, only the case $\sigma$ with $\abs{b} = 1$ is considered. We
    can assume $U$ is inside a ball with a suitable radius $r$ since $U$ is
    bounded subset in $\R^d$. Then this implies $\abs{c} \leq r$ as $\sigma(b\tr
    x + c)$ would be ones or zeros over all $U$. Suppose $b_0, b_1, c_0, c_1$
    and $\abs{b_0 - b_1} \leq \epsilon$ and $\abs{c_0 - c_1} \leq \epsilon$ for
    some $\epsilon < 0$. It is easy to check that 
    \begin{equation}
        \sup_{x\in U} \norm{\sigma({b_0, c_0}) - \sigma({b_1, c_1})}_2
            \lesssim \sqrt{\epsilon}
    \end{equation}
    in $\lp{2}$ with some constant independent of $d$.
    
    By the volume ratio argument \cite{vandervaartWeakConvergenceEmpirical1996},
    one can obtain a $\bigO(\sqrt{\epsilon})$-net for $A$ in $\lp{2}(U)$ if we
    are able to find a $\epsilon$-net for the set $P:= \{(b,c)\in \R^{d+1}:
    \abs{b} = 1, \abs{c} \leq r\}$. To build a $\epsilon$-net for the sphere
    $\abs{b}=1$, $\bigO(\epsilon^{1-d})$ elements is needed for the
    sphere $\{v \in \R^d, \abs{v}=1\}$. An interval $[-r,r]$ requires
    $\bigO(\epsilon^{-d})$ elements. Therefore, a $\epsilon$-net of
    $\bigO(\epsilon^{-2d})$ elements can be constructed for $A$ and hence the
    covering number for $A$ is of the order $\bigO(\epsilon^{-\frac{1}{2d}})$.

    The statement then follows the Corollary on~\cite[p.
    104]{makovozRandomApproximantsNeural1996}.

    % We denote by $D$ to avoid clutters.
    % \begin{equation}
    %     \Sigma = \{\sum_{j=1}^m a_j \indicator{b\tr x + c}, 
    %     \quad \sum_j \abs{a_i} \leq 1, \abs{b_j} = 1,
    %     b\in\R^d,
    %     a, c\in\R\}
    % \end{equation}
    % An estimate of entropy of $\Sigma$ w.r.t. the $\lp{q}$ norm.

    % It is sufficient to prove (10) only for the case s=_. Indeed, if s is an
    % arbitrary sigmoidal function, then s(*t) Ä _(t), * Ä +􏱶, uniformly on
    % every set |t|􏱵a>0; on [&a,a] the difference _(t)&s(*t) remains bounded. It
    % follows that &_v, b&sv, b&Lq(D) can be made arbitrarily small by taking a
    % sufficiently large &v&. For s=_ we can use (8) since obviously V=(A_)c. We
    % need an estimate for =n(A_). We may consider only the _v, b with |v|=1. If D
    % is contained in some ball |x|􏱴r, then we may assume that |b|􏱴r for
    % otherwise _v, b is identically 1 or 0 on D. Suppose that |v&v1|<=, |b&b1|<=
    % for some =>0. If v=v1 and, say, b>b1 , then _v, b &_v1, b1 is equal to \1 on
    % the strip &b􏱴vx􏱴&b1 of width 􏱴=, and to zero elsewhere. Similarly, if
    % b=b1, then _v,b&_v1,b1{0 only on a strip of width O(=). It follows that
    % &_v,b&_v1,b1&􏱴C-= in L2. (Here and below C stands for various con- stants
    % independent of n). Therefore we obtain an O(-=)-net for A_ in L2(D) if we
    % find an =-net for the set P :=[(v, b) # Rd+1 : |v|=1, |b|􏱴r]. By a standard
    % volume ratio argument, one needs O((1􏱩=)d&1) elements to build an =-net for
    % the sphere |v|=1 and O(1􏱩=) elements for the interval [&r, r], which gives
    % O(=&d ) elements for P. Consequently, one can find an =-net for A_ in L2
    % consisting of O(=&2d) elements. Thus =n(A_)= O(n&1􏱩(2d)), and (10) now
    % follows from (8).


\end{proof}

\section{Improved rate with higher smoothness index}

In the previous section, an error of $\bigO(n^{-1/2})$ is obtained for functions
in $\bspace{\mcal{F}, 1}$ using $n$ elements from the dictionary
\eqref{eq:dict_represent}
\begin{align}
    \label{eq:improved_fourier_dict}
    \mathbb{D} = \{
        \sigma(b\tr x + c), \quad b\in\R^d, c\in\R
    \}.
\end{align}

The smoothness of a function $f$ is expressed through its \textit{first} Fourier
representation and controlled via the spectral condition $v_{f,s}$
\eqref{eq:spectral_condition}. In particular, how ``oscillating'' or
``fluctuating'' of a function $f$ is measured by the mean of the norm of the
frequency vector weighted by the Fourier magnitude distribution. Naturally, we
would like to extend the findings with tighter restriction on the smoothness and
ideally decrease the error rate $\bigO(n^{-1/2})$. Tighter rates of
approximation is made possible with stricter conditions on the Barron spectral
norm while bounding the inner parameter $b\in\R^d$ in
\eqref{eq:improved_fourier_dict}.

It it shown recently in \cite{siegelSharpBoundsApproximation2022} that the
spectral norm $\norm{f}_{\bspace{\mcal{F}, s}}$, $s \geq 1$ is the variation
norm of the dictionary:
\begin{equation}
    \mathbb{D}_{\mcal{F}, s} := \{
        (1 + \abs{\omega})^{-s} e^{i\omega\tr x}: \omega \in \R^d
    \}.
\end{equation}

This implies that functions in Fourier-analytic Barron spaces $\bspace{\mcal{F},
s}$ can be approximated with $n$-term dictionary $\mathbb{D}_{\mcal{F}, s}$. We
define a set of finite linear combination of elements from
$\mathbb{D}_{\mcal{F},s}$ in which $\abs{a_j}$ is bounded in $\lp{1}$ for a
fixed $M > 0$
\begin{equation}
    \Sigma_n(\mathbb{D}_{\mcal{F}, s}, M) := \Bigg\{
        f = \sum_{j=1}^n \alpha_j d_j: 
        d_j \in \mathbb{D}_{\mcal{F}, s} \textnormal{ and } 
        \sum_{j=1}^n \abs{\alpha_j} \leq M, \quad 
        n \in \Nat, \alpha_j \in \R
    \Bigg\}.
\end{equation}


\subsection{Approximation in $\lp{\infty}$ with bounded coefficients}

\begin{theorem}
    Let $\Omega = [0,1]^d$ and $s > 0$. If $f$ is in the closed, symmetric
    convex hull of $\mathbb{D}_{\mcal{F}, s, M}$, then for a fixed $M<\infty$
    and any $n \in \Nat$, there exists a finite linear combination of elements
    $f_n \in \Sigma_n(\mathbb{D}_{\mcal{F}, s}, M)$ the approximation error in
    $\lp{\infty}(\Omega)$ is
    \begin{equation}
        \norm{f-f_n}_{\lp{\infty}(\Omega)} \lesssim 
        n^{-\frac{1}{2}-\frac{s}{d}} \sqrt{\log{n}} \norm{f}_{\bspace{\mcal{F},s}}.
    \end{equation}

\end{theorem}

\begin{proof}
    One can find the proof in
    \cite{klusowskiApproximationCombinationsReLU2018,siegelSharpBoundsApproximation2022}.
\end{proof}

\subsection{Approximation in $\lp{p}$ with bounded coefficients}

% A smoothness property of the function to be approximated is expressed in terms
% of its Fourier representation. In particular, an average of the norm of the
% frequency vector weighted by the Fourier magnitude distribution is used to
% measure the extent to which the function oscillates. In this Introduction, the
% result is presented in the case that the Fourier distribution has a density that
% is integrable as well as having a finite first moment. Somewhat greater
% generality is permitted in the theorem stated and proven in Sections III and IV.

% In addition to the smoothness property ensured via the finite first moment,


\section{Approximation with ReLU\textsuperscript{k} activation function}


%%% Local Variables: 
%%% mode: latex
%%% TeX-master: "MasterThesisSfS"
%%% End: 
