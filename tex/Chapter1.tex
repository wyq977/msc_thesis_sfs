\chapter{Nonlinear approximation}
\label{ch:preliminary}

In this chapter, we study the approximation results of. 
Write an overview based on \cite{devore_1998,pinkusApproximationTheoryMLP1999}.

\section{Preliminary results}
\label{sec:preliminary}

% \TONOTE{Is it Banach spaces or Hilbert space?}

% Jonas-Barron method

% Maurey's Theorem: with a closure of convex hull suggests bound

% Here I will start classical probabilistic argument of Maurey

In this section, we will introduce the approximation theory using $n$-terms from
a dictionary in a Hilbert space. This approximation problem is highly nonlinear
introduced by the selection of a dictionary. A dictionary could be an
arbitrarily chosen subset of $\mcal{H}$ but limits have to be imposed for
computational feasibility in practice. 

This section is mainly based on \cite[Chapter 8]{devore_1998} and
\cite{vandervaartWeakConvergenceEmpirical1996}, unless stated otherwise.

Let $\mathbb{D} = \{d_1,d_2,\dots\}$ be a uniformly bounded domain in a Hilbert
space $(\mcal{H}, \norm{\cdot}_{\mcal{H}})$
\begin{equation}
    \sup_{d_j\in \mathbb{D}} \norm{d_j}_{\mcal{H}} < \infty \quad 
    \forall d_j \in \mathbb{D}.
\end{equation}

For every $n \in \Nat$, the collection of all functions in $\mcal{H}$
which can be expressed as a linear combination of at most $n$ elements of
$\mathbb{D}$ (i.e. elements of \eqref{eq:gm}) is denoted by

\begin{equation}
    \label{eq:dict_represent}
    \Sigma_n(\mathbb{D}) = \Bigg\{
        \sum_{j=1}^n \alpha_j d_j, \quad
        d_j \in \mathbb{D}, \alpha_j \in \R
    \Bigg\}.
\end{equation}

Any function $f_n \in \Sigma_n(\mathbb{D})$ with $n \in \Nat$ is represented as
\begin{equation}
    \label{eq:gm}
    f_n = \sum_{j=1}^n \alpha_j d_j, \quad d_j \in \mathbb{D}.
\end{equation}

The problem of nonlinear approximation would generally be framed as: given a
target function $f \in \mcal{H}$, one needs to choose both a dictionary
$\mathbb{D}$ and $n$-term elements from $\mathbb{D}$ to approximate $f$. These
dictionaries \eqref{eq:dict_represent} can be interpreted as splines
approximation with free knots with a selected dictionary and we will study the
approximation results when $\mathbb{D}$ is comprised of sigmoidal functions and
ReLU.

To include the control over the coefficients, also called \textit{outer
parameters}, $\alpha_j$ in the above expansion, we introduce
\begin{equation}
    \label{eq:dict_sigma}
    \Sigma^{t}_n(\mathbb{D}, M) := \Bigg\{
        f = \sum_{j=1}^n \alpha_j d_j: 
        d_j \in \mathbb{D} \textnormal{ and } 
        \sum_{j=1}^n \abs{\alpha_j}^t \leq M^t, \quad 
        n \in \Nat, \alpha_j \in \R
    \Bigg\}
\end{equation}
for any $t \in \Nat \bigcup\, \{\infty\}$ and any $M > 0$. 

Let $K^t_n(\mathbb{D}, M)$ be the closure of $\Sigma^t_n(\mathbb{D}, M)$ in
$\mcal{H}$
\begin{equation}
    K^t_n(\mathbb{D}, M) := \closure{\Sigma^t_n(\mathbb{D}, M)}.
\end{equation}

When $t = 1$, $K^1_n(\mathbb{D}, M)$ is the class of functions that are a
\textit{convex} combination of elements in $\mathbb{D}$. When $t = \infty$,
$\Sigma^{\infty}_n(\mathbb{D}, M)$ corresponds to the sets whose coefficients
are bounded in $\lp{\infty}$. 

Next, we define the union for all $M > 0$
\begin{equation}
    K^t_n(\mathbb{D}) = \bigcup_{M > 0} K^t_n(\mathbb{D}, M).
\end{equation}

We define a seminorm for functions $f \in K^t(\mathbb{D})$
\begin{equation}
    \label{eq:general_seminorm}
    \abs{f}_{K^t(\mathbb{D})} = \inf \{
        M >0: f \in K^t_n(\mathbb{D}, M)
    \}.
\end{equation}

For $f \in \mcal{H}$, the approximation error is defined as
\begin{equation}
    \label{eq:appro_err_hilbert}
    e_n(f, \mathbb{D}, \mcal{H})
        := \sup_{g\in\Sigma_{\mathbb{D}}} \norm{f - g}_{\mcal{H}}.
\end{equation}

Assuming that approximation holds with a dictionary $\mathbb{D}$ in $\mcal{H}$
such that for any function $f \in \mcal{H}$, the approximation error is
\begin{equation}
    \label{eq:appro_error_general}
    e_n(f, \mathbb{D}, H) \leq n^{-\alpha} C_{\mathbb{D}}.
\end{equation}

We are concerned with the coefficients in the exponents $\alpha$ and what
$C_{\mathbb{D}}$ is dependent on.

For the special cases where $\mathbb{D}$ is the orthonormal basis of $\mcal{H}$,
the approximation error is given as
\begin{equation}
    \label{eq:appro_error_ortho}
    e_n(f, \mathbb{D}, \mcal{H}) 
        \lesssim n^{-\alpha} \abs{f}_{K^t(\mathbb{D})}.
\end{equation}

\begin{definition}[Covering numbers and entropy]
    \label{def:covering_num}
    Let $(\mcal{F}, \norm{\cdot}_{\mcal{F}})$ be a subset of a normed space. The
    \textit{covering number} $N(\epsilon, \mcal{F}, \norm{\cdot}_{\mcal{F}})$ is
    the minimal number of balls $\{g: \norm{f - g}_{\mcal{F}} < \epsilon\}$ of
    radius $\epsilon$ required to cover the set $\mcal{F}$. 
\end{definition}

\begin{definition}[Entropy]
    \label{def:entropy} The \textit{entropy} $\log N(\epsilon, \mcal{F},
    \norm{\cdot}_{\mcal{F}})$ is the logarithm of the covering number.
    \begin{equation}
        \log N(\epsilon, \mcal{F}, \norm{\cdot}_{\mcal{F}})
        = \inf \{
            \epsilon > 0: \mcal{F} \text{ is covered by } 2^n 
            \text{ balls of radius } \epsilon
        \}
    \end{equation}
\end{definition}

Approximation with a finite linear combination of elements from a bounded
dictionary in a Hilbert space $\mcal{H}$ achieves an error rate of
$\bigO(n^{-1/2})$ by~\cite{pisierRemarquesResultatNon1980} and this holds for
any bounded $\mathbb{D} \subset \mcal{H}$.

\begin{theorem}[Maurey's Theorem]
    \label{thm:maurey}
    Let $\mathbb{D} = \{d_1, d_2, \dots\}$ be a subset of a Hilbert space
    $(\mcal{H}, \norm{\cdot}_{\mcal{H}})$. $\mathbb{D}$ is uniformly bounded
    \begin{equation}
        \sup_{d\in\mathbb{D}} \norm{d}_{\mcal{H}} < \infty.
    \end{equation}
    For every $f \in \mcal{H}$ of the form
    \begin{equation}
        \label{eq:maurey_eq_1}
        f = \sum_{j=1}^m c_j d_j, \quad
        \sum_{j=1}^m \abs{c_i} < \infty, \quad
        c_j \in \R, m > 0
    \end{equation}
    and every $n \in \Nat$, there exists a $g_n = \sum_{j=1}^n a_j
    d_j$ with at most $n$ non-zero coefficients with
    \begin{equation}
        \sum_{j=1}^n \abs{a_j} \leq
        \sum_{j=1}^n \abs{c_j}, 
        \quad a_j \in \R
    \end{equation}
    such that
    \begin{equation}
        \norm{f - g_n}_{\mcal{H}} \leq
        2 \log N(\epsilon, \mathbb{D}, \norm{\cdot}_{\mcal{H}})
        \cdot n^{-\frac{1}{2}}
        \cdot \sum_{j=1}^n \abs{c_j}.
    \end{equation}

\end{theorem}

\begin{proof}
    Without loss of generality, we can assume $m < \infty$ and the coefficients
    $c_j$ in \eqref{eq:maurey_eq_1} are non-negative and the sum $\sum_{j=1}^m
    \abs{c_j} = 1$.

    We partition the index set $I = \{1,2,\cdots,m\}$ into $n$ nonempty subsets
    $I_p, \,p=1,2,\cdots,n$ such that each subset $\mathbb{D}_p$ of $\mathbb{D}$
    has a diameter smaller than $\epsilon$.

    For each partition $p = 1, 2,\cdots, n$, $n_p := \abs{I_p}$. Let $S_p = \sum_{i\in I_p} c_i$
    and 
    \begin{align}
        \hat{f}_p := \frac{S_p}{n_p} (\hat{d}_1^{(p)} + \cdots + \hat{d}_{n_p}^{(p)}) \\
        \hat{f} := \sum_{p=1}^n \hat{f}_p
    \end{align}
    where each $\hat{d}_{k}^{(p)}$ is identically distributed and pairwise
    independent from $\mathbb{D}_p$ and the probability of equalling to $d_i \in
    \mathbb{D}_p$ is $\frac{c_i}{S_p}$.
    \begin{equation}
        \PR{\hat{d}_{k}^{(p)} = d_i} = \frac{c_i}{S_p}, 
        \quad i \in I_p, p = 1,2,\cdots, n.
    \end{equation}

    Next, we can show that 
    \begin{equation}
        \ERW{\hat{f}_p} = \frac{S_p}{n_p} \sum_{k}^{n_p}\ERW{\hat{d}_{k}^{(p)}}
        = \frac{S_p}{n_p} \sum_{i\in I_p} \frac{c_i}{S_p} d_i = f_p
    \end{equation}

    \begin{equation}
        \ERW{\norm{f - \hat{f}}^2} = \sum^n_p \VAR{f_p - \hat{f}_p} = \sum^n_p \VAR{\hat{f}_p}
    \end{equation}

    As each partition $I_p$, the $\mathbb{D}_p$ has a diameter of smaller than
    $\epsilon$ which implies variance of $\hat{d}_p^{(p)}$ is controlled by
    $\epsilon^2$, we have
    \begin{equation}
        \ERW{\norm{f - \hat{f}}^2} =\sum^n_p \VAR{\hat{f}_p} \leq \sum^n_p \frac{\epsilon^2}{n} S_p = \epsilon^2/n.
    \end{equation}

    In conclusion, there must exist some realization $\hat{f}$ such that
    $\norm{f-\hat{f}}^2$ is smaller than $\epsilon^2/n$ and $\epsilon$ can be
    chosen close to the entropy. Such realization is a linear combination of at
    most $2n$ elements from $\mathbb{D}$.
\end{proof}

\begin{remark}
    This is refinement of the result in \cite{pisierRemarquesResultatNon1980} in a
    Hilbert space. 
\end{remark}

\cite{jonesSimpleLemmaGreedy1992} has also showed the same approximation error
$\bigO(n^{-1/2})$ using an iterative argument. 

Let $G$ be a nonempty bounded subset of a Hilbert space $\mcal{H}$. Suppose a
sequence of function $f_n$ were used to approximate a element $f$ in $\mcal{H}$
with the following algorithm
\begin{equation}
    \label{eq:seq_appro_jones}
    f_n = \alpha_n + f_{n-1} + (1-\alpha_n) g_n, \quad
    g_n \in G, \alpha_n \in [0,1], n \geq 1.
\end{equation}
The iteration starts with $f_1 = g_1 \in G$ and $\alpha_1 = 0$ and the sequence
$\{f_n\}$ defined as above is in the convex hull of the points $\{g_1,\dots,
g_n\}$.

\begin{theorem}[Jone's Theorem]
    Let $G$ be a nonempty bounded domain of a Hilbert space $\mcal{H}$. If $f$
    is in the closure of the convex hull of $G$, i.e. $f \in \conv(G)$. Then for
    every $n \in \Nat$, $f_n$ is chosen iteratively as
    \eqref{eq:seq_appro_jones}, the approximation error is $\bigO(n^{-1/2})$
\end{theorem}

\begin{proof}
    One can find a proof in \cite{jonesSimpleLemmaGreedy1992} and a refinement
    of Jone's result in \cite[Theorem
    5]{barronUniversalApproximationBounds1993}.    
\end{proof}

\begin{remark}
    The approximation rate by Maurey is not always sharp and one can use the
    compactness of the dictionary $\mathbb{D}$ to improve it.
\end{remark}


% Using a classical probabilistic argument of Maurey [45], an approximation rate
% of O(n 2 ) can be obtained for the class B1(D) using non-linear dictionary
% expansions. Moreover, Jones [27] gave a constructive proof of this fact using
% the relaxed greedy algorithm and applied this result to shallow neural networks
% with a cosine activation function. Improvements upon this rate of dictionary
% approximation under an assumption about the behavior of the relaxed greedy
% algorithm appear in [31, 33]. These results yield exponential rates of
% convergence for individual functions in the convex hull of D (but not
% necessarily its closure), which are however not uniform over the class B1(D).
% Further, under compactness [29, 41] or smoothness [50] assumptions on the
% dictionary D improved rates can also be obtained, although for general
% dictionaries the Maurey-Jones rate is the best one can expect [30].

\section{Universal approximation theorem}
\label{sec:uat}

In general, neural networks, even 2NN with one hidden layer, are universal
approximators, which means any continuous functions on a compact set can be
approximated up to a arbitrary precision under some mild restrictions on the
activation functions $\sigma$ and the number of nodes are allowed to grow
arbitrarily. Functions of $d$-variables can be approximated uniformly by linear
combinations of functions from the dictionary

\begin{equation}
    \label{eq:uat_dict}
    \mathbb{D} = \{\sigma(b\tr x + c): b \in \R^d, c\in \R \}.
\end{equation}

The requirements on $\sigma$ is formally stated
in~\cite{cybenkoApproximationSuperpositionsSigmoidal1989} and functions in
$\mathbb{D}$ are called \textit{ridge functions} or \textit{planar waves}.
Despite its obvious importance, one can not extract quantitative information
about the error rate since there is no restriction for $n$. Nevertheless, we
will state the univariate approximation theorem in this section. This section is
mainly based on \cite{cybenkoApproximationSuperpositionsSigmoidal1989}, unless
stated otherwise.

% We will state the the universal approximation theorem that the finite linear
% combinations of sigmoidal functions (sum of the functions of the form
% (\eqref{def:sigmoidal})) are dense in the space of continuous functions. This
% approximation result holds for any continuous sigmoidal functions.

% function on a compact set up to arbitrary precision [Cyb89, Fun89, HSW89,
% LLPS93]. modern mathematics of deep learning

\begin{definition}[sigmoidal function]\label{def:sigmoidal}
    A function $\sigma$ is \textbf{sigmoidal} if
    \begin{equation}
        \sigma(t) =
        \begin{cases}
            1 & \text{as} \quad t \to +\infty \\
            0 & \text{as} \quad t \to -\infty.
        \end{cases}
    \end{equation}
\end{definition}

\subsection*{Notations and setup}

Let $I_d = [0,1]^d$ denote the $d$-dimensional unit cube, and the space of
continuous functions over $I_d$ is denoted by $C(I_d)$. We denote the supremum
norm of a function $f \in C(I_d)$ by
\begin{equation}
    \label{def:sup_norm}
    \norm{f}_{\infty} = \sup_{x\in I_d} \abs{f(x)}.
\end{equation}
We use $M(I_d)$ to denote the space of finite, signed regular Borel measures,
i.e. $\mu(A) \in \R$ for all Borel sets $A \in I_d$ and $\mu(\emptyset)= 0$. We
refer readers to \cite{rudinFunctionalAnalysis1991,
rudinRealComplexAnalysis1987} for a detailed presentation of functional
construction used and we include some basic materials in Appendix
\ref{app:function_measure}.

Let $h(x)$ be a linear combinations of elements from the dictionary $\mathbb{D}$
\eqref{eq:uat_dict}

\begin{equation}
    \label{eq:sum_sigma}
    h(x) = \sum_{j=1}^n a_j d_j = \sum_{j=1}^n a_j \sigma(b_j\tr x + c_j)
\end{equation}

where $a_j, c_j \in \R$, $d_j \in D$,  $b_j \in \R^d$, $n < \infty$, and
$\sigma$ is a univariate function from $\R$ to $\R$.
 
The main contribution of approximation theorem is the statement on the
conditions of $\sigma$ such that the above finite linear combination $h(x)$ is
dense in $C(I_d)$ with respect to the supremum norm. It should also be stressed
that there is no restriction for the number of combinations.

\begin{theorem}[Universal approximation theorem]
    \label{thm:uat}
    If $\sigma$ is sigmoidal as defined in Definition $\ref{def:sigmoidal}$,
    then any function $f \in C(I_d)$ be approximated uniformly well by a finite
    linear combination of the form \eqref{eq:sum_sigma}.

    In other words, for any function $f: I_d \to \R$ and any $\epsilon > 0$.
    There exists a linear combination of $n$ ridge functions $d_j \in
    \mathbb{D}$ of the form \eqref{eq:uat_dict} such that
    \begin{align}
        f_n(x) = \sum_{j=1}^n a_j d_j, \quad d_j \in \mathbb{D}, n\in\Nat \\
        \norm{f - f_n}_{\infty} < \epsilon \quad \forall x \in I_d.
    \end{align}
\end{theorem}


The main structure of the proof is followed:
\begin{enumerate}
    \item any finite sums of the form \eqref{eq:sum_sigma} with a
    \hyperref[def:dis_func]{discriminatory function} $\sigma$ are dense in
    $C(I_d)$ with respective to the supremum norm;
    \item any bounded sigmoidal function is discriminatory.
\end{enumerate}

\begin{definition}[Discriminatory function]
    \label{def:dis_func}
    A function $\sigma$ is \textbf{discriminatory} if for every measure $\mu \in
    M(I_d)$,
    \begin{equation}
        \int_{I_d} \sigma(b\tr x + c) \,d\mu(x) = 0 \quad 
        \forall b \in \R^d \textnormal{ and } c \in \R
    \end{equation}
    implies $\mu = 0$.
\end{definition}

We continue to show that the linear span of any continuous discriminatory
functions are dense in the space of $(C(I_d), \norm{\cdot}_{\infty})$.

\begin{theorem}
    Let $\sigma: \R \to \R$ be a continuous
    \hyperref[def:dis_func]{discriminatory function}, the finite sums of the
    form \eqref{eq:sum_sigma} are dense in the space $(C(I_d),
    \norm{\cdot}_{\infty})$. In other words, for any $\epsilon > 0$ and any $f
    \in C(I_d)$, there exists a $n \in \Nat$ and a sum $h(x)$ of the above form,
    where
    \begin{equation}
        \norm{f - h}_{\infty} < \epsilon \quad \forall x \in I_d.
    \end{equation}
\end{theorem}

\begin{proof}
    Let $G := \spn(\{\sigma(b\tr x + c): b \in \R^d, c \in \R \})$ be the linear
    span for every $b\in\R^d, c\in\R$. $G$ clearly is a linear subspace of
    $C(I_d)$. We claim that the closure of $G$, $\closure{G}$, is all of
    $C(I_d)$.

    We continue the proof by contradiction. Assuming $\closure{G}$ is not
    $C(I_d)$, then there is a bounded linear functional $L$ on $C(I_d)$ such
    that $L \not\equiv 0$ on $C(I_d)$ and $L(G) = L(\closure{G}) = 0$ by the
    Hahn-Banach Theorem \ref{thm:hahn_banach_1}.

    By the \hyperref[thm:riesz_rep]{Riesz Representation Theorem}, there is a
    unique $\mu \in M(I_d)$ for this $L$ such that
    \begin{equation}
        L(f) = \int_{I_d} f(x) \,d\mu(x) \quad \forall f \in C(I_d)
    \end{equation}

    Since $L$ is identically zero on $G$, we must have for all $b$ and $c$ that
    \begin{equation}
        \int_{I_d} \sigma(b\tr x + c) \,d\mu(x) = 0
    \end{equation}

    However, the condition that $\sigma$ is discriminatory implies $\mu = 0$ and
    consequently $L = 0$. By \eqref{thm:hahn_banach_2}, subspace $G$ must be
    dense in $C(I_d)$.
\end{proof}

Now it remains to show that sigmoidal functions are discriminatory with the help
of Lemma \ref{lemma:zero_measure}.

\begin{lemma}~\cite{rudinFunctionalAnalysis1991}
    \label{lemma:zero_measure}
    if $\mu$ is a signed finite Borel measure on $\R^d$ such that the Fourier
    transform of $\mu$
    \begin{equation}
        \fourier{\mu}(u) = \int_{\R^d} e^{-iu\tr x} \,\mu(x) = 0,  
    \end{equation}
    for all $x\in\R^d$, then $\mu = 0$ for all measurable sets of $\R^d$.
\end{lemma}

\begin{lemma}
    Any bounded, measurable sigmoidal function is discriminatory.
\end{lemma}

\begin{proof}

    \textbf{Step 0} \textit{(Assume discriminatory and construct pointwise
    convergence function)}: Let $\sigma: \R \to \R$ be a sigmoidal function.
    Assume that the $\sigma$ is discriminatory with a measure $\mu \in M(I_d)$
    as in Definition \ref{def:dis_func} and the goal is to show that $\mu = 0$.

    Fix an arbitrary $b_0 \in \R^d\setminus \{0\}$ and define
    $\sigma_{\lambda}(x) := \sigma(\lambda (b_0\tr x + c) + \varphi)$. For
    any $c, \lambda, \varphi \in \R$, one can write
    \begin{equation}
        \sigma_{\lambda}(x)
        = \begin{cases}
            \to 1, \quad &\for b_0\tr x + c > 0 \quad \text{as}\, \lambda \to +\infty \\
            \to 0, \quad &\for b_0\tr x + c < 0 \quad \text{as}\, \lambda \to -\infty \\
            \sigma(\varphi), \quad &\for b_0\tr x + c = 0, \quad\forall \lambda \in \R
        \end{cases}.
    \end{equation}

    Therefore, the function $\sigma_{\lambda}$ converges pointwise to a function
    $\gamma(x): I_d \to \R$ as $\lambda \to + \infty$.

    \begin{equation}
        \gamma(x) = 
        \begin{cases}
            1,               \quad &\for b_0\tr x + c > 0 \\
            0,               \quad &\for b_0\tr x + c < 0 \\
            \sigma(\varphi), \quad &\for b_0\tr x + c = 0
        \end{cases}
    \end{equation}
    
    Let $\Pi_{b_0,c}$ denote the hyperplane and $H_{b_0, c}$ denote the
    half-space as: 
    \begin{align}
        \Pi_{b_0,c} &= \{x\in\R^d \mid b_0\tr x + c = 0\} \\
        H_{b_0, c}  &= \{x\in\R^d \mid b_0\tr x + c > 0\}
    \end{align}
    for all $c \in \R$.
    
    By the Lebesgue Convergence Theorem, we have
    \begin{align*}
        \sigma(\varphi) \mu(\Pi_{b_0, c}) + \mu (H_{b_0, c})
        &= \int_{I_d} \gamma(x) \,d\mu(x) \\
        &= \int_{I_d} \lim_{\lambda\to\infty} \sigma_{\lambda}(x)\,d\mu(x) \\
        &= \lim_{\lambda\to\infty} \int_{I_d} \sigma_{\lambda}(x)\,d\mu(x) = 0
    \end{align*}
    for all $\lambda, \varphi \in \R$.

    Thanks to the function $\sigma$ being sigmoidal, we have $\lim_{\varphi\to
    +\infty} \sigma(\varphi)= 1$ and $\lim_{\varphi\to -\infty}
    \sigma(\varphi)=0$ and consequently it is easy to see
    \begin{equation}
        \label{eq:disc_1}
        \mu(\Pi_{b_0, c}) = 0 \quad \text{and} \quad \mu (H_{b_0, c}) = 0 
        \quad \forall c\in\R.
    \end{equation}

    This in turn implies $\mu(I_d) = 0$ as $c$ can be chosen arbitrarily large.
    
    We would like to show that the measure of all half-planes being zero implies
    that the measure $\mu$ must be zero. If $\mu$ is a positive Borel measure,
    this would be trivial by \eqref{eq:disc_1} but $\mu$ here is a signed
    measure.
    
    \textbf{Step 1} \textit{(Construct a signed measure)}: Let $\phi$ be a
    finite signed, Borel measure on $\R$
    \begin{equation}
        \label{eq:disc_2}
        \phi(A) = \mu(\{x \in I_d: b_o\tr x \in A\}) \quad \forall A \subseteq \R.
    \end{equation}

    By construction, we have
    \begin{align}
        \forall a < b \in \R, \quad \phi((a,b)) 
        &= \phi((a,\infty)) - \phi([b, \infty)) \\
        &= \mu(\{x \in I_d: b_o\tr x > a\}) \\
        &\quad- \Big(
            \mu(\{x \in I_d: b_o\tr x > b\})
            + \mu(\{x \in I_d: b_o\tr x = b\})
        \Big) \\
        &= \mu(H_{b_0, -a}) - (\mu(H_{b_0, -b}) + \mu(\Pi_{b_0, -b})) \\
        &= 0 - 0 = 0.
    \end{align}
    Therefore $\phi(A) = 0$ for all Borel sets $A \subseteq \R$.

    % Show a bounded linear functional is zero on such measure
    \textbf{Step 2} \textit{(Define a linear functional $L$)}: Let
    $\lp{\infty}(\R)$ denote the space of all measurable bounded functions $f:
    \R \to \R$. For a function $h \in \lp{\infty}(\R)$, we define a functional
    $L: \lp{\infty}(\R) \to \R$:
    \begin{equation}
        L(h) = \int_{I_d} h(b\tr x)\,d\mu(x).
    \end{equation}

    $L$ is linear because for all $\alpha, \beta \in \R, \text{ and } g, h \in
    \lp{\infty}(\R)$
    \begin{align*}
        L(\alpha g + \beta h)
        &= \int_{I_d} (\alpha g + \beta h) (b_0\tr x) \,d\mu(x) \\
        &= \int_{I_d} (\alpha g(b_0\tr x) + \beta h (b_0\tr x)) \, d\mu(x) \\
        &= \alpha \int_{I_d} g(b_0\tr x) \,d\mu(x) 
            + \beta \int_{I_d} h(b_0\tr x) \,d\mu(x) \\
        &= \alpha F(g) + \beta L(h)
    \end{align*}

    Now we look at the indicator function for all Borel sets of $\R$
    \begin{equation}
        \indicator{A}(x) =
        \begin{cases}
            1, \quad \text{ if } x \in A, \\
            0, \quad \text{ if } x \not\in A.
        \end{cases}
    \end{equation}

    Combining $\indicator{A} \in \lp{\infty}(\R)$ and \eqref{eq:disc_2}, we
    have 

    \begin{equation}
        L(\indicator{A}) = \int_{I_d} \indicator{A}(b_0\tr x)\,d\mu(x)
        = \mu(\{ x\in I_d: b_0\tr x\in A \}) = \phi(A) = 0
    \end{equation}
    for all Borel sets $A \subseteq \R$.

    Since $L$ is a linear functional, the finite sum of functions of the form
    (simple functions):
    \begin{equation}
        \label{eq:simple_function}
        s_n(x) = \sum_{j=1}^{n} a_j \indicator{A_j}(x)
    \end{equation}
    is zero for all $n\in\Nat$ where $a_j \in \R$, and ${A_j} \subseteq \R$ are
    measurable and pairwise disjoint sets of $\R$. Since the simple functions
    \eqref{eq:simple_function} are dense in $\lp{\infty}(\R)$, then for a
    function $h \in \lp{\infty}(\R)$, there exists a sequence of functions $s_n$
    converge pointwise to $h$ for all $n \in \Nat$ and $x \in \R$
    \begin{align}
        L(h) &= \int_{I_d} h(b_0\tr x) \,d\mu(x) 
             = \int_{I_d} \lim_{n\to\infty} s_n(b_0\tr x) \,d\mu(x) \\
             &= \lim_{n\to\infty} \int_{I_d} s_n(b_0\tr x) \,d\mu(x)
             = \lim_{n\to\infty} L(s_n)
             = 0
    \end{align}
    where the limit in the integral is moved outside the integral by the
    Lebesgue Convergence Theorem.

    \textbf{Step 3} \textit{(Tidy up)}: sine and cosine functions are in
    $\lp{\infty}(\R)$, which implies $L(\cos) = L(\sin) = 0$ for all $b_0$. For a
    bounded measurable function $h'(x) = e^{ib_0\tr x} = \cos(b_0\tr x) + i
    \sin(b_0\tr x)$, we have
    \begin{equation}
        \label{eq:bounded_functional}
        L(h') = \int_{I_d} h'(b_0\tr x) \,d\mu(x) 
              = \int_{I_d} \cos(b_0\tr x) + i \sin(b_0\tr x) \,d\mu(x)
              = 0.
    \end{equation}

    It is easy to verify that $\mu = 0$ on $\R^d \setminus \{0\}$ as the Fourier
    transform of $\mu$ is zero from Lemma \ref{lemma:zero_measure}. Combining
    with the fact that $\mu(I_d) = 0$, \eqref{eq:bounded_functional} establishes
    that $\sigma$ is discriminatory.

\end{proof}

% Let $f_m(\mathbf{x}, \Theta)$ be the parameterized family of 2NN defined above
% of $m$ nodes that which map input vector $\mathbf{x}$ of dimension $d$.

% \begin{equation}
%     f_m(\mathbf{x}, \Theta) = \sum_{i=1}^m a_i \sigma(\mathbf{b_i}^T\mathbf{x} + c_i)
% \end{equation}

% where $\Theta = (a_1, \mathbf{b}_1, c_1, \dots, a_m, \mathbf{b}_m, c_m)$ denotes
% all the parameters (the total number of parameters is $(d+2)m + 1$). We will
% consider the hypothesis spaces where the activation functions $\sigma$ is ReLU
% \footnote{\TONOTE{The choice of activation function is important for
%         infinite-width Barron spaces}}.

%  homogeneity

\subsection{Application to the classification problem}

This section will explicate the implications of Theorem \ref{thm:uat} for
classification problems. It should be noted that the decision function defined
below is not continuous on $I_d$. We would like to check whether such
classification problem can also be well understood as the regression problem.

\begin{definition}[Decision function]
    Let $\{P_1, \dots, P_k\}$ be a partition of $I_d$ where each partition $P_j$
    are pairwise disjoint nonempty Borel sets, i.e. $P_j \not= \emptyset$ and
    $\bigcup_{j=1}^k P_j = I_d$. $f$ is a decision function for $I_d$ of the
    form
    \begin{equation}
        f: I_d \to \{1, \dots, k\}
    \end{equation}
    where $f(x) = j$ for $x \in P_j$. 
\end{definition}

\begin{theorem}
    \label{thm:uat_clas}
    Let $\sigma$ be a continuous sigmoidal function and $f$ be the decision
    function for any finite, measurable partition of $I_d$. Let $\phi$ be a
    Borel measure on $I_d$ and $\mu(I_d) = 1$. Then for any $\epsilon>0$, there
    exists a finite sum of the form
    \begin{equation}
        G(x) = \sum_{j=1}^n a_j \sigma(b_j\tr x + c_j)
    \end{equation}
    where $a_j, c_j \in \R, b_j \in \R^d$ and a set $D\subset I_d$, such that
    the measure of the set $D$, $\mu(D) \geq 1 - \epsilon$ and
    \begin{equation}
        \sup_{x\in D}\abs{G(x) - f(x)} < \epsilon
    \end{equation}
\end{theorem}

Theorem \ref{thm:uat_clas} is a anoloy of the UAT and the proof is
straightforward using Lusin's Theorem \eqref{thm:lusin}.

\begin{proof}
    By Lusin's Theorem \eqref{thm:lusin}, there exists a continuous function $g
    \in C(I_d)$ such that $\mu(\{x\in I_d \mid f(x) \not= g(x)\}) < \epsilon$.
    Now we have a continuous $g$ and we are able to a find a sum of the form
    above satisfying $\abs{G(x) - g(x)} < \epsilon$ by Theorem \eqref{thm:uat}
    for all $x\in I_d$. Let set $D = \{x\in I_d \mid f(x) = g(x)\}$ and we have
    $\mu(D) \geq 1 - \epsilon$. Then for $x\in D$, we have
    \begin{equation}
        \sup_{x\in D}\abs{G(x)-f(x)} = \sup_{x\in D}\abs{G(x) - g(x)} < \epsilon 
        \quad \forall x\in D.
    \end{equation}
\end{proof}

The above result shows that the total measure of the misclassified points can be
made arbitrarily small when $n$ is allowed to grow.


%%% Local Variables: 
%%% mode: latex
%%% TeX-master: "MasterThesisSfS"
%%% End: 
