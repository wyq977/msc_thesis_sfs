\chapter{The approximation properties of two-layer neural networks}

In the previous chapter, we introduced the basic concepts of the two-layer
neural networks (\gls{2nn}). Section \ref{sec:preliminary} introduces the
approximation theory using $n$-term dictionary in a Hilbert space. In Section
\ref{sec:uat}, we will state the result of the universal approximation theorem.
Later in Section \ref{sec:spectral_norm} and \ref{sec:different_barron_spaces},
we formally introduce the function spaces associated with 2NN: the
\textit{Fourier-analytic Barron spaces} and \textit{infinite-width Barron
spaces}. The norms associated with these spaces (\textit{spectral norm},
\textit{Barron norm}) are examined in Section \ref{sec:different_barron_spaces}.
An error rate of $\bigO(n^{-1/2})$ for functions in these spaces are given in
Section \ref{sec:spectral_norm} and \ref{sec:barron_norm}. This chapter aims to
provide a concise summary of various well-known results concerning the
approximation properties of 2NN, and we recommend work by
\cite{eMathematicalUnderstandingNeural2020,bernerModernMathematicsDeep2021} for
a comprehensive review.

% classical Barron space, or the Fourier-analytic Barron space

% infinitely wide

% infinite-width Barron spaces.

% sparsity-inducing norm ? in bach 2017 paper.


\section{Different Barron spaces}
\label{sec:different_barron_spaces}

The notation of Barron spaces is not in consensus within the community and
different terms have been given to describe the same model classes or spaces.
For the function spaces in which functions have finite Fourier moments,
~\cite{xuFiniteNeuronMethod2020} call this model classes \textit{Barron spectral
spaces} while \cite{carageaNeuralNetworkApproximation2022} refers them as
\textit{Fourier-analytic Barron space}.  For function spaces in which functions
admit an integral representation \eqref{eq:barron_represent} with a ReLU
activation function, \cite{eBarronSpaceFlowinduced2021} refer them simply as
\textit{Barron spaces} of different orders $p \in \Nat \bigcup\,\{\infty\}$. The
term \textit{Barron space} was coined by \cite{ePrioriEstimatesPopulation2019}
to honor Prof. Andrew Barron's contribution in the understanding of neural nets.
In some literature including~\cite{carageaNeuralNetworkApproximation2022}, these
spaces are named \textit{infinite-width Barron spaces} associated with different
activation functions and the term \textit{classical Barron space} is reserved
for those associated with Heaviside function\footnote{Also called step function
or unit step function}.

To avoid confusion and in the meantime emphasize Prof. Andrew Barron's
contribution, we will use two definitions:
\begin{itemize}
    \item Fourier-analytic Barron spaces
    \item infinite-width Barron spaces\footnote{
        We limit the model classes to those associated with ReLU only. In the 
        case of Heaviside function, we denote the space still by the term 
        \hyperref[def:heaviside_space]{\textit{classical Barron space}}.
    }
\end{itemize}

In general, the activation function associated with the infinite-width Barron
spaces is ReLU and we will explicitly state when other functions (e.g. squared
ReLU, ReLU\textsuperscript{k}, Heaviside) are used.



\subsection{infinite-width Barron spaces}
\label{sec:barron_norm}

This section introduces the infinite-width Barron space and its elementary
properties, which is mostly based on \cite{eBarronSpaceFlowinduced2021} unless
stated otherwise.

Let $U$ be a nonempty and bounded domain in $\R^d$. For functions $f: U \to
\mathbb{R}$, we consider those that admit the following integral representation:

\begin{equation}
    \label{eq:barron_represent}
    f(x) = \int_{\Omega} a \sigma(b\tr x + c) \mu(da, db, dc), \quad 
    x \in U, a,c \in \R, b \in \R^d.
\end{equation}

Let $\Omega = \R^1 \times \R^d \times \R^1$ and $\Sigma_{\Omega}$ be the
$\sigma$-algebra on $\Omega$. Define an integral condition for $f$ w.r.t. a
measure $\mu \in \Sigma_{\Omega}$
\begin{align}
    r(f, \mu, p)
    &= \Big(\int_{\R^d} \abs{a}^p  (\norm{b}_1 + \abs{c})^p \,d\mu(a,b,c)\Big)^{1/p} \\
    &= \Big(\ERWi{\mu}{\abs{a}^p  (\norm{b}_1 + \abs{c})^p}\Big)^{1/p},
    \quad 1 \leq p \leq +\infty.
\end{align}

$\mu$ is a probability distribution on $(\Omega, \Sigma_\Omega)$, $\Omega =
\mathbb{R}^1 \times \mathbb{R}^d \times \mathbb{R}^1$ and $\Sigma_\Omega$ is a
Borel $\sigma$-algebra on $\Omega$ and $\sigma(\cdot)$ is the ReLU activation
function.

We consider another case where the ReLU function is replaced by the Heaviside
function.

\begin{definition}[Heaviside function]
    \label{eq:heaviside_represent}
    \begin{equation}
        H(x) = 
        \begin{cases}
            1 \quad x > 0,\\
            0 \quad x \leq 0    
        \end{cases}
    \end{equation}
\end{definition}


Similarly, for a $\mu \in \Sigma_{\Omega}$, we consider functions $f$ that admit
the representation below
\begin{equation}
    \label{eq:heaviside_represent}
    f(x) = \int_{\Omega} a H(b\tr x + c) \mu(da, db, dc), \quad x \in U.
\end{equation}

Accordingly, we define a condition for that particular $\mu$ where
\eqref{eq:heaviside_represent} holds
\begin{align}
    r(f,\mu, H)
    = \int_{\Omega} \abs{a}\,d\mu(a,b,c) = \ERWi{\mu}{\abs{a}}
\end{align}

These representations can be seen as a continuum analogy of the 2NN with $b$
hidden nodes:

\begin{equation}
    f_n(x, \Theta) := \frac{1}{n}
    \sum_{j=1}^n a_j 
        \sigma(b\tr x + c_j), 
    \quad \Theta = \{(a_j, b_j, c_j), \ j = 1, \dots, n\}.
\end{equation}

\begin{definition}[Barron norm] For a function $f$ that admits the integral
    representation in \eqref{eq:barron_represent}, the Barron norm is defined as:
    \begin{equation}\label{eq:barron_norm}
        \barronnorm{f}{p} := \inf_{\rho} \Big(\ERWi{\mu}{\abs{a}^p 
        (\norm{b}_1 + \abs{c})^p}\Big)^{1/p},
        \quad 1 \leq p \leq +\infty.
    \end{equation}
    % where \begin{equation*} \Theta_f = \left\{ (a, \pi) \mid f(x) =
    % \int_{space} a(w) \sigma(\langle w, x \rangle)\dd{\pi(w)} \right\}.
    % \end{equation*}
\end{definition}

The infimum is taken over all probability distribution where
\eqref{eq:barron_represent} holds for all $x \in U$. When $p = + \infty$, the
Barron norm reads

\begin{equation}
    \label{eq:barron_infinite_norm}
    \barronnorm{f}{\infty} :=
    \inf_{\rho} \max_{a, b, c \in \supp(\rho)} \abs{a} (\norm{b}_1 + \abs{c}).
\end{equation}

Similarly, we can define a norm associated with Heaviside function where the
infimum is taken for all measure $\mu$ where \eqref{eq:heaviside_represent}
holds
\begin{equation}
    \label{def:heaviside_norm}
    \norm{f}_{\mcal{B}_H} = \inf_{\rho} \ERWi{\mu}{\abs{a}},
    \quad 1 \leq p \leq +\infty.
\end{equation}

\begin{definition}[Infinite-width Barron space]
    \label{def:barron_space}
    Let $U$ be a nonempty bounded domain in $\R^d$. For functions that admit
    representation \eqref{eq:barron_represent}, the infinite-width Barron space
    with an order of $1 \leq p \leq \infty$ is
    \begin{equation}
        \mcal{B}_p(U) = \Bigg\{
            f: U \to \R : \exists\, \mu \in \Sigma_{\Omega}: 
            r(f, \mu, p) < \infty \textnormal{ and }
            \forall x \in U, f(x) = \int_{\Omega} a \sigma(b\tr x + c) \mu(da, db, dc)
        \Bigg\}.
    \end{equation}
\end{definition}

A normed space can be defined for those associated with Heaviside function.

\begin{definition}[Classical Barron space]
    \label{def:heaviside_space}
    Let $U$ be a nonempty unbounded domain in $\R^d$. For functions that admit
    representation \eqref{eq:barron_represent}, the infinite-width Barron space
    associated with Heaviside function is
    \begin{equation}
        \label{def:heaviside_space}
        \mcal{B}_H(U) = \Bigg\{
            f: U \to \R : \exists\, \mu \in \Sigma_{\Omega}:
            r(f, \mu, H) < \infty \textnormal{ and }
            \forall x \in U, f(x) = \int_{\Omega} a H(b\tr x + c) \mu(da, db, dc)
        \Bigg\}.
    \end{equation}
\end{definition}

Barron spaces are denoted by $\mathcal{B}_p$~\footnote{
    Going forward, we will simplify $\mcal{B}_{\mcal{F}, s}(U)$, $\mcal{B}_p(U)$ 
    and $\mcal{B}_H(U)$ as $\mcal{B}_{\mcal{F}, S}$, $\mcal{B}_p$ and 
    $\mcal{B}_H$ to avoid cluttering the notations when $U$ is a bounded domain
    in $\R^d$.
}, consist of all the functions whose $r(f, \mu, p)$ is finite for a measure 
$\mu \in \Sigma_{\Omega}$.

\begin{proposition}
    By the definition of Barron norm, it is easy to see that

    \begin{equation}
        \mathcal{B}_{\infty} \subset \cdots \subset \mathcal{B}_{2} 
        \subset \mathcal{B}_1.
    \end{equation}
\end{proposition}


% https://ocw.mit.edu/courses/18-125-measure-and-integration-fall-2003/6f21af6c40de1eccd70349bd3a3b0095_18125_lec17.pdf

\begin{proof}

The idea is similar to the inclusion of $L_p$, $L_q$ space.

Applying Hölder's inequality, for any $1 \leq p \leq q < \infty$

\begin{align*}
    \int \abs{a}^p (\norm{b}_1 + \abs{c})^p d\rho
     & = \int \abs{a}^p (\norm{b}_1 + \abs{c})^p \cdot 1 d\rho                                                    \\
     & \leq \Big(\int \abs{a}^{pq/p} (\norm{b}_1 + \abs{c})^{pq/p} d\rho \Big)^{p/q} \Big(\int d\rho\Big)^{1-p/q} \\
     & = \Big(\int \abs{a}^q (\norm{b}_1 + \abs{c})^q d\rho \Big)^{p/q} \Big(\int d\rho\Big)^{1-p/q}
\end{align*}

Therefore we have the inclusion $\mathcal{B}_{q} \subset \mathcal{B}_p$ for $1
    \leq p \leq q \leq \infty$.
\end{proof}

As the reverse also holds in the class of ReLU functions,  we have
$\mathcal{B}_{\infty} = \mathcal{B}_p$, $\barronnorm{\cdot}{\infty} =
    \barronnorm{\cdot}{p}$  for all $1 \leq p \leq +\infty$.

% ~\cite[Proposition 1]{eBarronSpaceFlowinduced2021}
\begin{proposition}
    \label{lamma:equivalence_barron_space}

    For any $f \in \mathcal{B}_1$, $f
        \,\text{also}\, \in \mathcal{B}_{\infty}$ and $\barronnorm{f}{\infty} =
        \barronnorm{f}{p}$ and hence $ \mathcal{B}_{\infty} = \cdots =
        \mathcal{B}_{2} = \mathcal{B}_1$ when $\sigma(\cdot)$ is ReLU function.
\end{proposition}


\section{Approximation in infinite-width Barron spaces}
\label{sec:appro_barron_space}

\begin{theorem} [Direct Approximation in $\lp{2}$]\
    \label{thm:barron_direct_appro_l2}
    Let $U$ be a nonempty unbounded domain in $[0,1]^d$ and $f: U \to \R$. For
    any function $f \in \mathcal{B}(U)$ and any integer $n \in \Nat$, there
    exists a 2NN $f_n = f(x, \Theta) = \frac{1}{n}\sum_{j=1}^n a_j \sigma(b_j\tr
    x + c_j)$ such that
    \begin{equation}
        \norm{f - f_n}_2 \lesssim \frac{\norm{f}_{\mathcal{B}}}{\sqrt{n}}.
    \end{equation}
    where $\Theta$ is the set of parameters $\Theta = \{(a_j, b_j, c_j),
    j=1,\dots,m\}$.

    % Furthermore, we have
    % \begin{equation}
    %     \norm{\Theta}_{\textnormal{path}} 
    %     := \frac{1}{n} \sum_{j=1}^m \abs{a_j} 
    %     (
    %         \norm{b_j}_1 + \abs{c_j}
    %     )
    %     \leq 3\norm{f}_{\mathcal{B}}
    % \end{equation}
\end{theorem}

% The $\norm{\Theta}_{\textnormal{path}}$ is the per-unit norm control for 2NN
% defined in \cite{neyshaburNormBasedCapacityControl2015}. 

\begin{theorem}[Direct approximation in $\lp{\infty}$]
    Given the conditions outlined in Theorem \ref{thm:barron_direct_appro_l2},
    it can be established that for any function $f \in \mcal{B}(U)$ and any
    integer $n \in \Nat$, there exists a 2NN such that
    \begin{equation}
        \norm{f - f_n}_{\infty} \lesssim 
        \norm{f}_{\mathcal{B}} \sqrt{\frac{d+1}{n}}.
    \end{equation}
\end{theorem}


\section{Difference and Connection Between Different Barron Spaces}
\label{sec:diff_barron_spaces}

This section will clarify the relationships between  Barron spaces, namely the
\textit{Fourier-analytic Barron spaces} and the \textit{infinite-width Barron
spaces}. Although some relationships between these spaces has been examined and
understood partially in
\cite{eBarronSpaceFlowinduced2021,eMathematicalUnderstandingNeural2020}, we hope
to clarify this problem in this section inspirerd by the work from
\cite{carageaNeuralNetworkApproximation2022}.

Firstly, Let $\Sigma_{\Omega}$ denote the set of all Borel probability
measures on $\Omega = \R^1 \times \R^d \times \R^1$ and we write functions that
admits the integral form
\begin{equation}
    \label{eq:integral_represent}
    f(x) = \int_{\omega} a\sigma(b\tr x + c) \,d\mu(a,b,c), \quad
    \forall x \in \R^d.
\end{equation}

Given a nonempty, bounded domain $U \subset \R^d$, we have already defined
various Barron spaces:

\begin{itemize}
    \item $\bspace{\mcal{F}, s}$, $s \in \{1,2\}$ Fourier-analytic Barron spaces 
        \eqref{def:fourier_space} with $\norm{\cdot}_{\mcal{F},s}$ 
        \eqref{def:spectral_norm}
    \item $\mcal{B}$ infinite-width Barron spaces \eqref{def:barron_space} with
        $\norm{\cdot}_{\mcal{B}}$ \eqref{eq:barron_norm}
    {
        \setlength\itemindent{25pt}
        \item $\bspace{H}$ classical Barron space \eqref{def:heaviside_space}
            with $\norm{\cdot}_{\mcal{B}_H}$ \eqref{def:heaviside_norm}
    }
\end{itemize}

We \textit{could} include the classical Barron space within the infinite-width
Barron spaces when $\bspace{p}(U), p=0$ with a ReLU$^0$ activation function
which is essentially the Heaviside function but we decide against it to
emphasize on $\bspace{H}(U)$ as it is frequently visited. We denote
infinite-width Barron spaces with $\bspace{}(U)$ after the equivalence of
$\bspace{p}(U), 1\leq p \leq\infty$ has been shown in Proposition
\ref{lamma:equivalence_barron_space}.

% hookrightarrow or subset
\begin{lemma}
    Given the constructions of spaces above and a nonempty bounded domain $U$ in
    $\R^d$, then the following relationships holds:

    1) $\mcal{B}(U) \subset \bspace{H}(U)$

    2) $\bspace{\mathcal{F}, 1}(U) \subset \bspace{H}(U)$

    3) $\bspace{\mathcal{F}, 2}(U) \subset \mcal{B}(U)$
\end{lemma}


\begin{proof}

% As shown in weinan, functions in the $\mathcal{B}_{ReLU}$ i.e. Barron space, are
% Lipschitz.

% https://math.stackexchange.com/questions/2319301/why-must-bounded-sets-be-contained-within-a-closed-ball
\textbf{1):} 
One can begin with the connection between the ReLU and the Heaviside function.
As $U$ is nonempty and bounded in $\R^d$, there is a open ball for some $x \in
\R^d$, $B_r(\cdot)$, with a radius $\delta > 0$ whose closure contains $U$ such
that
\begin{equation}
    U \subset \closure{B_{\delta}(x)}.
\end{equation}

For $x=0$ and a suitable $\delta$, $U \subset \closure{B_{\delta}(0)}$ then we
have:

\begin{equation}
    \sigma(x) = \int_0^{1+\delta} H(x-t) \,dt 
    \quad \forall x \in \R \textnormal{ and } \abs{x} < \delta.
\end{equation}

Let $\beta_{b,c}$ be $\abs{b} + \abs{c}$ for any $b\in\R^d, c\in\R$. It is easy
to see that $\abs{b\tr x + c} \leq (1+\delta)\beta_{b,c}$. Thanks to the
positive homogeneity of ReLU function $\sigma$, i.e. $\sigma(\lambda x) =
\lambda \sigma(x)$ for $x \in \R$, we observe that any function $f:U\to\R$ that
admits such an integral representation with a measure $\mu \in \Sigma_{\Omega}$
can be rewritten as
\begin{align*}
    f(x) 
    &= \int_{\Omega} a \sigma(b\tr x + c )\,d\mu(a,b,c) \\
    &= \int_{\Omega} \beta_{b,c} \sigma(
        \frac{b\tr x}{\beta_{b,c}}+ \frac{c}{\beta_{b,c}}
    ) \,d\mu(a,b,c) \\
    &= \int_{\Omega} \int_0^{1+\delta} 
        a\beta_{b,c} H(\frac{b\tr x}{\beta_{b,c}} +
        \frac{c}{\beta_{b,c}} -t)\,dt\,d\mu(a,b,c) \quad
        \text{(Fubini's Theorem)} \\
    &= \int_{\Omega} a' H(b\tr x + c') \,d\nu(a',b,c'), 
    \quad \forall x \in U
\end{align*}
where $a', c' \in \R$ for some $v \in \Sigma_{\Omega}$.

The inclusion is immediate if one can find a measure $v$ and the integral
condition $r(f, v, H)$ is also finite.

With a mapping
\begin{equation}
    T: \Omega \times [0,1+\delta] \to \Omega, \quad 
    ((a,b,c), t) \mapsto
    (a\beta_{b,c}, \frac{b}{\beta_{b,c}}, \frac{c}{\beta_{b,c}} - t)
\end{equation}

 we can construct the measure $v$ via the pushforward of the product measure
$\mu\otimes\lambda$, given $\lambda$ is the Lebesgue measure on the interval
$[0,1+\delta]$,
\begin{equation}
    v := T^{-1}(\mu\otimes\lambda).
\end{equation}

Furthermore, we can evaluate the $r(f, v, H)$
\begin{align}
    r(f, v, H) 
    &= \int_{\Omega} \abs{a} \,dv(a,v,c) 
    = \int_{\Omega}\int_0^{1+\delta} \abs{a\beta_{b,c}} \,dt\,d\mu(a,b,c) \\
    &= (1+\delta) \abs{a}(\abs{b} + \abs{c}) \,d\mu(a,b,c)
    = (1+\delta) r(f,\mu) < \infty.
\end{align}

Therefore, it shows that for any function $f \in \mcal{B}(U)$
\begin{equation}
    \norm{f}_{\mathcal{B}_H} \lesssim \norm{f}_{\mcal{B}} < \infty
\end{equation}
hence the inclusion holds.

\textbf{2):} This is a direct consequence of \cite[Theorem
2]{barronNeuralNetApproximation1992}.

% and we include the proof for completeness.

% \begin{theorem}
%     Let $f$ be a function that admits a Fourier representation with finite
%     spectral norm of order $1$, i.e.
%     \begin{equation}
%         v_{f,1} < \infty
%     \end{equation}
%     then $f(x) - f(0)$ can be expressed as an infinite convex combination of
%     indicator multiplied by a constant
%     \begin{equation}
%         f(x) - f(0) = \int_{\R^d}\int_0^1 (
%             \indicator{} - \indicator{}
%         ) \,d\omega\,dt
%     \end{equation}
% \end{theorem}

% By Fourier transform, note that
% \begin{equation}
%     f(x) - f(0) = \int (e^{i\omega\tr x} - 1) \fourier{f}(\omega)\,d\omega
% \end{equation}

% We also have
% \begin{equation}
%     e^{iz} - 1 =
%     \begin{cases}
%         i \int_0^c \indicator{\{z>u\}} e^{iu} \,du \quad
%             &\text{ when } z \in [0,c] \\
%         -i \int_0^c \indicator{\{z<-u\}} e^{iu} \,du \quad
%             &\text{ when } z \in [c,0] 
%     \end{cases}
% \end{equation}

% It follows that
% \begin{equation}
%     e^{iz} - 1 = i \int_0^c (
%         \indicator{\{z>u\}} - \indicator{\{z<-u\}}  
%     )
%     e^{iu} \,du 
% \end{equation}

% Integrating when $z = \omega x$ and $c=\abs{\omega}_{[0,1]}$ defined in
% \eqref{def:fourier_class} gives
% \begin{equation}
%     f(x) -f(0) = i\int_{\R^d} (\int_0^{c} (
%         \indicator{\{\omega x>u\}} - \indicator{\{\omega x<-u\}}
%     ) e^{iu}\,du)
%     \fourier{f}(\omega)\,d\omega
% \end{equation}

% Only taking the real part of the LHS and RHS and integrating using Fubini's theorem
% \begin{equation}
%     f(x) = f(0) + \int_{\R^d} \int_0^1 (
%         \indicator{\{\omega x<-t\}} - \indicator{\{\omega x>t\}}
%     ) \abs{\omega} \sin(t\omega + \theta_{\omega})
%     \fourier{f}(\omega)\,d\omega\,d\omega
% \end{equation}

% A similar argument as the proof in \TONOTE{proof of Fourier} is employed here
% where the ``drawing'' parameters from the distribution shows that the $f(x) -
% f(0)$ is the closure of the convex hull of finite linear combinations.

% Hence we show the inclusion in 2).

\textbf{3):} As $U$ is nonempty and bounded in $\R^d$, we can fix a point $x_0
\in \R^d$ and a radius $\delta > 0$ such that $U \subset x_0 + [0,\delta]^d$.
Without loss of generality, it is safe to assume that $f$ is a function in
$\mathcal{B}_{\mathcal{F},2}$ with the spectral condition $v_{f,2}\leq 1$
\eqref{def:spectral_seminorm}. This implies \hyperref[def:spectral_norm]{spetral
norm} $\norm{f}_{\bspace{\mcal{F}, 2}} \leq 2$ by direct calculation. 

One can prove the inclusion if $f$ can be represented as in
\eqref{eq:integral_represent} with a measure in $\Sigma_{\Omega}$.

We define two mapping $G, H: \R^d \to \mathbb{C}$:
\begin{align*}
    G(\omega) &= \frac{1}{2} (\fourier{f}(\omega) 
                    + \overline{\fourier{f}(-\omega)}) \\
    H(\omega) &= \frac{1}{\delta^d} \cdot 
                    e^{\frac{i \omega\tr x_0}{\delta}} \cdot 
                    G(\omega / \delta)
\end{align*}
where $\fourier{f}$ is the Fourier transform of $f$ and $\overline{\fourier{f}}$
is the complex conjugate of $\fourier{f}$.

We calculate their respective spectral norm in $\bspace{\mcal{F}, 2}$:
\begin{align}
    \norm{G}_{\mathcal{F},2} &\leq 2 \\
    \norm{H}_{\mathcal{F},2} &\leq 2\delta^2
\end{align}

We then define two functions from $U$ to $\R$ with $G,H$ as their Fourier
transform, respectively.
\begin{align*}
    g(x) &:= \int_{\R^d} e^{i\omega\tr x} G(\omega)\,d\omega\\
    h(x) &:= \int_{\R^d} e^{i\omega\tr x} H(\omega)\,d\omega\\
\end{align*}
It is easy to check that for all $x \in U$, $f(x)=g(x)=h(\frac{x-x_0}{\delta})$.

By construction, the spectral condition of $h$, $v_{h,2}$, is finite. We have
$\norm{h}_{\mathcal{B}}$ is finite with some constant $C_h$ thanks to Theorem 9
in \cite{eMathematicalUnderstandingNeural2020}. Therefore, $h(y)$ can be
represented as
\begin{equation}
    h(y) = \int_{[0,1]^d} a \sigma(b\tr x + c) \,d\mu(a,b,c) \quad
    \forall y \in [0,1]^d
\end{equation}
where $\mu \in \Sigma_{\Omega}$ and $\norm{f}_{\mcal{B}} < \infty$ w.r.t. some
constant only dependent on $\delta$ and $d$.

Since $y=\frac{x-x_0}{\delta}$ for all $x\in U$, the results above implies that 
\begin{equation}
    f(x) = h(\frac{x-x_0}{\delta}) \int_{\Omega} a 
    \sigma(\frac{b\tr x}{\delta} + c - \frac{b\tr x_0}{\delta})\,dv(a,b,c)
\end{equation}
for some measure $v \in \Sigma_{\Omega}$.

We continue the construction of measure via the pushforward of $v = T(\mu)$. Let
$T$ be a mapping:
\begin{equation}
    T: \Omega \to \Omega, \quad
    (a,b,c) \mapsto (a, \frac{b}{\delta}, c - \frac{b\tr x_0}{\delta})
\end{equation}

By calculation, $r(f, v) \leq (1+\abs{x_0}) r(h, \mu)$ is finite w.r.t some
constant $C$ dependent only on $d,\delta,x_0$. Hence the inclusion is shown.

\end{proof}


\begin{proposition}

    Let $U \subset \mathbb{R}^d$ be bounded and have nonempty interior. For
    $s>0$, if $\mathcal{B}_{\mathcal{F},s}(U) \subset \mathcal{B}_{1}(U)$, then
    $s \geq 2$. In particular $\mathcal{B}_{\mathcal{F},1}(U) \not\subset
    \mathcal{B}_{1}(U)$.
\end{proposition}

\begin{remark}

    It has been argued in \cite{eRepresentationFormulasPointwise2020} that
    $\bspace{\mcal{F},1}$ embeds into the $\mcal{B}$ but
    \cite{carageaNeuralNetworkApproximation2022} proven that this embedment
    wrongly interpretes the results of
    \cite{barronUniversalApproximationBounds1993,barronNeuralNetApproximation1992}.
    In other words, the original model class proposed by
    \cite{barronNeuralNetApproximation1992} is not \textit{contained} in the
    novel Barron space $\mcal{B}$ introduced recently by
    \cite{eBarronSpaceFlowinduced2021}. 
    
    % It has been discussed in earlier section that we have the equivalence of
    % Barron norm and Barron spaces for $1 \leq p\leq +\infty$ (see
    % \eqref{eq:barron_fouier_int}) with ReLU as activation function.

    One can observe that the class of functions in which $f$ admits a Fourier
    representation with finite second moment, $\bspace{\mcal{F}, 2}$, is well
    ``contained'' inside the \textit{infinite-width Barron space}. However,
    $\mathcal{B}_{\mathcal{F}, 1}$ still encapsulates a boarder class of
    functions. In other words, the infinite-width Barron space $\mcal{B}$ is
    \textit{sandwiched} between $\bspace{\mcal{F}, 1}$ and $\bspace{\mcal{F},
    2}$.
\end{remark}


\section{Improved approximation from $\bigO(n^{-1/2})$}
\label{sec:improvement}


In this section, we present an exposition on the improved approximation error
rates. The curx of the notion is to derive the entropy number corresponding to
the convex hull of the dictionary $\mathbb{D}$. Maurey's Theorem
\ref{thm:maurey} provides a bound for \textit{any} bounded dictionaries in a
Hilbert space. The improvement was first done
by~\cite{makovozRandomApproximantsNeural1996} and subsequently shown by
\cite{klusowskiApproximationCombinationsReLU2018} where ReLU and squared ReLU
are used as the activation functions with norm controlled parameters. The former
improvement is made possible through the selection of the Heaviside function,
and a superior rate $\bigO(n^{-1/2 - 1/q\cdot d})$ was obtained in Section
\ref{subsec:improved_heaviside}. In the latter, the \textit{inner parameters}
$b\in\R^d$ in \eqref{eq:improved_fourier_dict} are suitably constrained to
ensure compactness. The works of \cite{maUniformApproximationRates2022,
siegelSharpBoundsApproximation2022, klusowskiApproximationCombinationsReLU2018a}
provide the basis for our discussion, unless explicitly cited otherwise.

\subsection{Fourier-analytic Barron spaces with higher smoothness index}

In the previous section, an error of $\bigO(n^{-1/2})$ is obtained for functions
in $\bspace{\mcal{F}, 1}$ using $n$ elements from the dictionary
\eqref{eq:dict_represent}
\begin{align}
    \label{eq:improved_fourier_dict}
    \mathbb{D} = \{
        \sigma(b\tr x + c), \quad b\in\R^d, c\in\R
    \}.
\end{align}

The smoothness of a function $f$ is expressed through its \textit{first} Fourier
representation and controlled via the spectral condition $v_{f,s}$
\eqref{eq:spectral_condition}. In particular, how ``oscillating'' or
``fluctuating'' of a function $f$ is measured by the mean of the norm of the
frequency vector weighted by the Fourier magnitude distribution. Naturally, we
would like to extend the findings with tighter restriction on the smoothness and
ideally decrease the error rate $\bigO(n^{-1/2})$. Tighter rates of
approximation is made possible with stricter conditions on the Barron spectral
norm while bounding the inner parameter $b\in\R^d$ in
\eqref{eq:improved_fourier_dict}.

It it shown recently in \cite{siegelSharpBoundsApproximation2022} that the
spectral norm $\norm{f}_{\bspace{\mcal{F}, s}}$, $s \geq 1$ is the variation
norm of the dictionary:
\begin{equation}
    \mathbb{D}_{\mcal{F}, s} := \{
        (1 + \abs{\omega})^{-s} e^{i\omega\tr x}: \omega \in \R^d
    \}.
\end{equation}

This implies that functions in Fourier-analytic Barron spaces $\bspace{\mcal{F},
s}$ can be approximated with $n$-term dictionary $\mathbb{D}_{\mcal{F}, s}$. We
define a set of finite linear combination of elements from
$\mathbb{D}_{\mcal{F},s}$ in which $\abs{a_j}$ is bounded in $\lp{1}$
\begin{equation}
    \Sigma_n(\mathbb{D}_{\mcal{F}, s}, M) := \Bigg\{
        f = \sum_{j=1}^n \alpha_j d_j: 
        d_j \in \mathbb{D}_{\mcal{F}, s} \textnormal{ and } 
        \sum_{j=1}^n \abs{\alpha_j} \leq M, \quad 
        n \in \Nat, \alpha_j \in \R
    \Bigg\}.
\end{equation}


\textbf{Approximation in $\lp{\infty}$ with bounded coefficients}

\begin{theorem}
    Let $\Omega = [0,1]^d$ and $s > 0$. If $f$ is in the closed, symmetric
    convex hull of $\mathbb{D}_{\mcal{F}, s, M}$, then for a fixed $M<\infty$
    and any $n \in \Nat$, there exists a finite linear combination of elements
    from $\Sigma_n(\mathbb{D}_{\mcal{F}, s}, M)$ the approximation error in
    $\lp{\infty}(\Omega)$ is
    \begin{equation}
        \sup_{x\in \Omega} \norm{f(x)-f_n(x)}_{\infty} \lesssim 
        n^{-\frac{1}{2}-\frac{s}{d}} \sqrt{\log{n}} \norm{f}_{\bspace{\mcal{F},s}}.
    \end{equation}

    In other words,
    \begin{equation}
        \inf_{f\in \Sigma_n(\mathbb{D}_{\mcal{F}, s}, M)} 
        \norm{f-f_n}_{\infty} \lesssim 
        n^{-\frac{1}{2}-\frac{s}{d}} \sqrt{\log{n}} \norm{f}_{\bspace{\mcal{F},s}}.
    \end{equation}
\end{theorem}

\begin{proof}
    One can find the proof in
    \cite{klusowskiApproximationCombinationsReLU2018,siegelSharpBoundsApproximation2022}.
\end{proof}

\textbf{Approximation in $\lp{q}$ with bounded coefficients}

\subsection{Fourier-analytic Barron spaces with ReLU\textsuperscript{k}}

In this section, we consider approximation by 2NN with ReLU$^k$ as activation
function
\begin{equation}
    \sigma_k(x) = [\max(0, x)]^k, \quad k \in \Nat^0
\end{equation}
when $k = 0$, $\sigma_0$ is the Heaviside function in Definition
\ref{eq:heaviside_represent}.

% A smoothness property of the function to be approximated is expressed in terms
% of its Fourier representation. In particular, an average of the norm of the
% frequency vector weighted by the Fourier magnitude distribution is used to
% measure the extent to which the function oscillates. In this Introduction, the
% result is presented in the case that the Fourier distribution has a density that
% is integrable as well as having a finite first moment. Somewhat greater
% generality is permitted in the theorem stated and proven in Sections III and IV.

% In addition to the smoothness property ensured via the finite first moment,

The dictionaries of interest corresponding to the $\sigma_k$ activation function
\begin{equation}
    \mathbb{D}_{k} = \Bigg\{
        \sigma_k(b\tr x + c), \quad b\in S^{d-1}, c\in [c_1, c_2] 
    \Bigg\}
\end{equation}
where $\sigma_k$ is ReLU$^k$ function described above, $S^{d-1} := \{x\in\R^d:
\norm{x}_2 = 1\}$ is the unit sphere in $\R^d$. $c$ is chosen such that
\begin{equation}
    c_1 < \inf_{x\in\Omega} b\tr x < 
    \sup_{x\in\Omega} b\tr x < c_2, \quad
    b \in S^{d-1}
\end{equation}

We define the sets of functions whose coefficients $a_j$ are bounded in $\lp{1}$
and $\lp{\infty}$ for all $0 < n < \infty$.

\begin{align*}
    \Sigma_n^1(\mathbb{D}_k, M) &:= \Big\{
        \sum_{j=1}^n a_j d_j:
        d_j \in D^d_{\sigma_k}, \sum_{j=1}^n \norm{a_j}_1 \leq M 
    \Big\} \\
    \Sigma_n^{\infty}(\mathbb{D}_k, M) &:= \Big\{
        \sum_{j=1}^n a_j d_j:
        d_j \in D^d_{\sigma_k}, \max_{j} \abs{a_j} \leq M 
    \Big\}
\end{align*}

We will consider the cases where the $\lp{1}$-norm of $a_i$ is bounded or
unbounded.
% https://e.math.cornell.edu/people/belk/measuretheory/LpFunctions.pdf

\TONOTE{Theorems on the improved rates with its conditions}

\TONOTE{A summary table for all rates}



% \subsection{Approximation in $L_{\infty}$ with bounded weights}

% We start with the main result when the approximation with 2NN with $m$ nodes and 
% the parameters bounded in $L_1$.

% \begin{theorem}
%     \label{thm:appro_bound_l1}
%     Let $U = [-1,1]^d$. Suppose $f: U \to \R$ admits a Fourier representation
%     and the spectral norm of order $2$ of $f$ is finite, i.e.
%     \begin{equation}
%         v_{f,2} = \int_{\R^d} \norm{\omega}_1^s \abs{\fourier{f}(\omega)} 
%         d\omega < \infty.
%     \end{equation}
%     There exists a 2NN of the form with ReLU activation function $\sigma$
%     \begin{equation}
%         f_n(x) = f(0) + \nabla f(0) \cdot x + v \cdot 
%         \frac{1}{n} \sum_{j=1}^n a_j \sigma(b_j\tr x + c_j)
%     \end{equation}
%     with $a_j\in[-1,1]$, $\norm{b_j}_1 = 1$, $c_j\in[0,1]$ and $v \leq
%     2v_{f,2}$ such that
%     \begin{equation}
%         \sup_{\mathbf{x} \in D} \norm{f(x) - f_m(x)}_{\infty} 
%         \leq c v_{f,2} \sqrt{d+\log{n}} \, n^{-1/2-1/d}
%     \end{equation}

%     for some universal $c > 0$.
% \end{theorem}

% \begin{theorem}
%     Let $U = [-1,1]^d$. Suppose $f: U \to \R$ admits a Fourier representation
%     and the spectral norm of order $2$ of $f$ is finite, i.e.
%     \begin{equation}
%         v_{f,3} = \int_{\R^d} \norm{\omega}_1^3 \abs{\fourier{f}(\omega)} 
%         d\omega < \infty.
%     \end{equation}
%     There exists a 2NN of the form with squared ReLU activation function
%     \begin{equation}
%         f_n(x) = f(0) + \nabla f(0) \cdot x + v \cdot 
%         \frac{1}{n} \sum_{j=1}^n a_j \sigma(b_j\tr x + c_j)
%     \end{equation}
%     with $a_j\in[-1,1]$, $\norm{b_j}_1 = 1$, $c_j\in[0,1]$ and $v \leq
%     2v_{f,2}$ such that
%     \begin{equation}
%         \sup_{\mathbf{x} \in D} \abs{f(x) - f_m(x)} \leq 
%         c v_{f,3} \sqrt{d} n^{-1/2-1/d}
%     \end{equation}

%     for some universal $c > 0$.
% \end{theorem}


 
% \section{Minimax Lower Bounds for Two Neural Network Model}

% For simplicity, data are of the form $\{(U_i, Y_i)\}_{i=1}^n$ from a

% For functions $f$ in the class $\mathcal{F}: [-1, 1]^d \to \mathcal{R}$, the
% minimax risk is:

% \begin{equation} R_{n,d} := \inf_{\hat{f}} \sup_{f\in \mathcal{F}}
%     \ERW{\norm{f - \hat{f}}^2} \end{equation}

% It has been shown in \TOCITE(Barron Minimax paper) that the minimax lower
% bound for neural nets is of order $\bigO(\log{d}/n)$ to some fractional power
% when $d$ is of larger order than $n$.

% \section{Risk Bounds for }

% {\itshape Firstly, I need to understand the difference between approximation
% rates and estimation of population risk.

% There exists a vast dictionary of definitions with similar wordings but the
% context is hughly different.

% e.g. What's the estimate of the population risk? In my understanding,

% $f^* - f_m$ where $f_m$ is the best solution in such hypothesis class is the
% approximation rates

% It should be that Population risk is w.r.t. a particular loss function?}

% \begin{equation} f(x) = \sum_{i=1}^m a_i \phi (\spr{\mathbf{w_i}}{x})
%     \end{equation}





%%% Local Variables: %% mode: latex %% TeX-master: "MasterThesisSfS" %% End: 
