\chapter{Meetings Log}

\section{25.10.2022}

\subsection{Questions Asked}

\textbf{Problems with Implicit or Explicit Regularization in ~\cite{ePrioriEstimatesPopulation2019}}
The problem is w.r.t. to the explicit regularization by considering the
so-called path norm with parameter $\lambda$. If it's just 2NN with GD or SGD,
assuming random init., then it should be implicit regularization by the
\textit{intrinsic} properties of 2NN structure or the optimizers.

\textbf{Why it might not have been possible to use inequaility from Empirical process to prove the main theorem in ~\cite{barronUniversalApproximationBounds1993}}
Assuming we are following the idea of the proof in original paper, considering the 
space $G_{step}$ is actually a lot bigger. Instead of discussing the VC of spaces of half planes
it's talking about the convex hall of all the half planes or step functions so the inequality 
may not be suited.

\textbf{``Induced by measure $\mu$''}
This is talking about $x \in \mathcal{R}^d$ with measure $\mu$ 
(whether it's finite or infinite? Les or just a measure: Not known) \
and hence the transformation $z = \alpha x \in [0, 1]$ has a another measure (?)

\textbf{Difference between the representation theorem and direct approximation theorem ~\cite{parhiBanachSpaceRepresenter2021}}
The representation theorem involves a map from $\mathcal{R}^N$ to $\mathcal{R}$, which is simply 
the loss functions (where $N$ is actually refering the number of points or sample size)
Its statement is that such optimization problem would have a solution that is 
in the \textbf{\textit{exact}} form of so called Barron function with a measure in 
$\Omega = \{\text{space of the parameters}\}$

On the other hand, Inverse or Direct approximation theorem is more about:
if a $f$ in said space, then it can be represented with a finite width (how large is also important, but more related to the rate)
of 2NN with a more or less ``good'' error.

\textbf{Why would later in 2016 2018 paper, they focus on the different $s$ in the spectral norm and its rate}
with $s$ increase, the class of functions become smaller. (\tonote{why}) 
Are they trying to get a smaller near optimal rate?

\subsection{Unsolved}

\begin{itemize}
    \item Still unsure about the $T_{\alpha}$ induced by the measure
    \item Need to gather all the rates
    \item But first, need to check whethere they are all talking about approximation rates
\end{itemize}


\section{01.11.2022}

\subsection{Question asked}

\textbf{Difference in Notation of Barron norm by E in 2 papers}

In \cite{ePrioriEstimatesPopulation2019}, function has the integral representation form like:

\begin{equation*}
    f(x) = \int_{\mathbb{S}^d} a(w) \sigma(\spr{w}{x}) d\pi(x)
\end{equation*}

and the norm for Barron $p$, $\Theta_f$ is the function spaces with the above representation

\begin{equation*}
    \gamma_p(f) := \inf_{(a, \pi \in \Theta_f)} (\int_{\mathbb{S}^d} \abs{a(w)} d\pi(x))^{1/p}
\end{equation*}

No idea why $\mathbb{S}^d$ need to be defined: $\mathbb{S}^d = \{w \mid \norm{w}_1 = 1\}$

The connection with Barron Spectral Norm (Fourier norm):
\begin{equation*}
    \gamma_{\infty}(f) := \sup_{w\in \mathbf{S}^d} \abs{a(w)} \lesssim \int_{\mathbf{S}^d} \norm{\omega}_1^2 \abs{\hat{f}(\omega)} d\omega
\end{equation*}

While in \cite{eBarronSpaceFlowinduced2021}, the integral representation:
\begin{equation*}
    f(x) = \int_{\Omega} a \sigma(\mathbf{b}^T\mathbf{x} + c) \rho(da, db, dc)
\end{equation*}

And the Barron norm
\begin{align*}
    \barronnorm{f}{p} = \inf_{\rho} (\ERWi{\rho}{\abs{a}^p(\norm{\mathbf{b}}_1 + \abs{c}^p)})^{1/p} \\
    \barronnorm{f}{\infty} = \inf_{\rho} \max_{(a,\mathbf{b},c \in \supp{\rho})} \abs{a}(\norm{\mathbf{b}}_1 + \abs{c})
\end{align*}

What's the support of $\rho$ with $\rho$ being the measure in the integral representation?

Whats the $a(w)$ in the first version

\textbf{What's the difference between the rate ?}

\textbf{\cite{parhiRoleNeuralNetwork2020}}

\textbf{\cite{heissHowImplicitRegularization2021}}

\newpage

\section{23.11.2022}

{\small bar is the complex conjugation}

\textbf{Outline for Intro/Pre-requisite:}
\begin{itemize}
    \item Note on measure theory
    \item spline theorey, spline regression
    \item Integral transform, mostly on Fourier Transform, Parseval’s Theorem \\
    Mainly on $v_{f,s}$ or $C_{f,s}$, the norm of frequency vector weighted by Fourier magnitude distribution. \\
    Fourier moment and the smoothness of the functions with couter-example
    \item The idea of random sampling strategy in the analysis (heavily used in Barron)
    \item Covering number/net, VC-dim (cardinality-Based) (used in some proofs),
    \item \tonote{Parameterized dictionary of in-finite dim}
    \item Risk bound (Estimation)
    \item ERM ?
\end{itemize}


\textbf{Outline for the approximation properties}
\begin{itemize}
    \item Universal approximation theorem by Cybenko, Hornik
    \item Approximation rate/bounds with finite
    \begin{itemize}
        \item Approximation theorem
        \item Different Risk bounds, in \cite{barronApproximationEstimationBounds1994, klusowskiRiskBoundsHighdimensional2018}
        \item Properties of Barron functions and good functions (finite Fourier moments)
    \end{itemize}
    \item Finite Fourier-analytic norm
    \begin{itemize}
        \item Approximation bound (1/m) Estimation bound (log n  md/n )
        \item Risk bounds for different penalized estimator \\
        1994 (Complexity penalty) 2018 (pen and find a universal c) \\
        $\ERW{\norm{T\hat{f} - f^*}^2} \leq c \inf\{\norm{f - f^*}^2 + \ERW{pen_n(f)/n]}\}$
    \end{itemize}
    \item Infinite-width Barron norm \\
    Inverse and Direct approximation theorem \\
    Risk bound with path norm \tonote{Should it be in explicit section?}
    \item Relationship with spaces that arise from the norms above mentioned (inclusion )
\end{itemize}

\textbf{Outline for explicit regularization}

\textbf{Outline for implicit regularization}
\begin{itemize}
    \item Showcase of implicit works, empirical evidence/experiments \\
    ``Let us add here that it was shown in [44] that functions of the form (1.3) arise naturally as the solutions of appropriately regularized learning problems.''
    \item connection with smoothing spline regression by Nowak
    \item smoothing spline regression solution (with early stopping)
    \item Is the path norm also implicit?
    \item weight decay, ..., etc.
\end{itemize}


\newpage
$d$: input data dimension

$m$: number of parameters (number of nodes in this case)

$n$: number of sample

Here we are talking about the distance between the function and the 
penalized least sqaures estimator (in Barron's paper) or greedily obtained.


If the function f has a Fourier representation, then the spectral norm
$v_{f, s} = \int \norm{\omega}_1^s \tilde{f}(\omega) d\omega$

1. $\bigO(\frac{v_{f,1}^2}{m}) + \bigO(\frac{md}{n})\log{n}$ by \cite[Theorem 3/4]{barronApproximationEstimationBounds1994}

Assumption: 

Regularization:

2. a spectral norm times $\bigO((\frac{\log{d}}{n})^{1/3})$ by \cite[Theorem 1]{klusowskiRiskBoundsHighdimensional2018}

Approximation results: for function $f$ with finite $v_{f,2}$ or $v_{f_3}$, there exists 2NN, bounded by $\frac{v_{f,2}^2}{m}$ and $\frac{v_{f,3}^2}{m}$ 

Assumption: 

Regularization/Penalty:

3. 


\newpage

\section{08.02.2022}

\subsection{Outline for implicit bias/regularization}

\begin{itemize}
    \item Prevalence of "regularization/bias" + Numerical/empirical description
    of this phenomena
    \item Gradient descent "favors": min norm or max margin solution w/o
    explicit regularization
    \item learn NN "almost" interpolates samples
    \item qualitative work: mainly on low-Complexity ones, mainly in the context
    of regression and mostly on infinitely wide network.
    \item specificly, infinitely-wide 2NN with l2 bounded weights.
    \item quantitative: not so many...
    \item Role of learning rate: Even 1 step learning obtains representation
    learned \\ High-dimensional Asymptotics of Feature Learning: How One Gradient
    Step Improves the Representation.
    \item Function spaces associated with shallow NN, (bounded norm)
    \item 
\end{itemize}

\subsection*{Convex hull results for the approximation}

The structure and the result are very similar to one and other.

Knowing the $N(T, \epsilon)$, then the $\log N(conv(T), \epsilon)$.

Not sure about the connection and Can't really make at the moment.

\subsection*{admin}

being of Mar: gives a detailed chapter before regularization 

mid Mar: draft/final for opinion




%%% Local Variables: 
%%% mode: latex
%%% TeX-master: "MasterThesisSfS"
%%% End: 