\chapter{Functional Analysis}
\label{app:function_measure}

\numberwithin{equation}{chapter}


We will briefly review some of the constructions and theorems of functional
analysis and measure theory used throughout this paper. Further background
material can be found in
\cite{rudinFunctionalAnalysis1991,rudinRealComplexAnalysis1987}. Familiarity
with the basic notions of topology is assumed (inner product spaces, normed
spaces, Banach and Hilbert spaces).

% https://www.math.uzh.ch/gorodnik/functional_analysis/lecture4.pdf

% convex hull
% http://staff.ustc.edu.cn/~wangzuoq/Courses/15F-FA/Notes/FA16.pdf

Let $X$ be any vector space, and $K \subset X$ a subset.

\begin{definition}[Convex hull]
   \label{def:convex_hull}
   The \textit{convex hull} of $E$ is the set:
   \begin{equation}
      \conv(E) 
      = \{x = a_1x_1 + \cdots + a_nx_n \in X \mid 
      x_1, \cdots, x_n \in E \quad
      t_1 + \cdots t_n = 1, t_i \geq 0\}
   \end{equation}
   and an element $x$ in $\conv(E)$ is called a convex combination of $x_1,
   \cdots, x_n$.
\end{definition}

From the definition, it is easy to see that 
\begin{itemize}
   \item $E \subset \conv(E)$, $\conv(E)$ is convex.
   \item If a set $A \subset X$ is convex and $E \subset A$, then $\conv(E) \subset A$.
\end{itemize}

Since any intersection of convex sets is still convex, we can get the following
equivalent definition of convex hull.

\begin{proposition}
   The convex hull of a set $E$ is 
   \begin{equation}
      \conv(E) = \bigcap_{E \subset A, A \text{ is convex}} A.
   \end{equation}   
\end{proposition}

\begin{definition}[Closed convex hull]
   \label{def:closed_convex_hull}
   The closure of the convex hull of $E$ is called closed convex hull of $E$,
   denoted by $\closure{\conv(E)}$.
\end{definition}

It is easy to see that $\closure{\conv(E)}$ is a closed convex set, and it is
the smallest closed convex set containing $E$. Another way to view it is that
$\closure{\conv(E))}$ is the intersection of all closed convex sets that
contains $E$:
\begin{equation}
   \closure{\conv(E)} = \bigcap_{E \subset A, A \text{ is convex and closed}} A   
\end{equation}

We have the following for $K \subseteq E$ in $X$, $\closure{\conv(K)} \subseteq
\closure{\conv(E)}$.





\begin{definition}[Linear opertor]
   Let $X$ and $Y$ be normed spaces over $\R$\footnote{
       It can be extended to $\mathbb{C}$
   }. $T: X \to Y$ is a linear operator if $T$ is linear, i.e.
   \begin{align*}
      T(x+y)       &= T(x) + T(y) \quad \forall x \in X, y \in Y \\
      T(\lambda x) &= \lambda T(x) \quad \forall x \in X, \lambda \in \Nat
   \end{align*}
\end{definition}

\begin{definition}[Bounded linear operator]
   Let $X$ and $Y$ be normed spaces. A linear operator $T: X \to Y$ is bounded
   if there exist a $M > 0$ such that for all $x \in X$,
   \begin{equation*}
      \norm{T(x)}_{Y} \leq M \norm{x}_{X}
   \end{equation*}
\end{definition}

A linear operator between normed spaces is bounded if and only if it is
continuous. We use $B(X, Y)$ to denote the space of Bounded operator between $X$
and $Y$.

\begin{definition}[Linear functional]
   Let $X$ be a normed space over $\R$ and $T: X \to \R$ is a linear operator,
   then $T$ is a linear functional on $X$.
\end{definition}

A linear functional $T$ is \textit{bounded} if and only if there exists a $\lambda >
0$ such that $T(x) \leq \lambda \norm{x}_{X}$ for all $x \in X$.

\begin{definition}[Dual space]
   The dual space of normed space $X$ is the vector space $X^*$ whose elements
   are the continuous linear functionals on $X$. In other words, $X^* = B(X,
   \R)$.
\end{definition}

\begin{theorem}[Consequence II of Hahn-Banach Theorem]
   \label{thm:hahn_banach_1}
   Let $(X, \norm{\cdot}_{X})$ be a subspace of $(Y, \norm{\cdot}_{Y})$ and $x_0
   \in X$. $x_0$ is in the closure $\closure{X}$ of $X$ if and only if there is
   no bounded linear functional $T: Y \to \R$ on $(Y, \norm{\cdot}_{Y})$ such
   that $T(x) = 0$ for all $x \in X$ while $F(x_0) \not= 0$
\end{theorem}

% https://www.imsc.res.in/~kesh/hahn.pdf

\begin{corollary}[Consequence II of Hahn-Banach Theorem]
   \label{thm:hahn_banach_2}
   Let $X$ be subspace of a normed linear space $Y$. Assume $\closure{X}$, the
   closure of $X$, is not $Y$. Then there exists a bounded linear functional $L$
   on $Y$ such that $L \not\equiv 0$ and $L(x) = 0$ for all $x \in X$. Suppose
   every bounded functional $L$ on $Y$ is identically zero on $Y$. Then $X$ is
   dense in $Y$.
\end{corollary}

\begin{theorem}[Riesz Representation Theorem]
   \cite[Theorem 6.19, p. 130]{rudinRealComplexAnalysis1987}
   \label{thm:riesz_rep}

   If $X$ is a locally compact Hausdorff space, then every bounded linear
   functional $L$ on the the space of all continuous functions on $X$, $C(X)$,
   is represented by a unique regular complex Borel measure $\mu$
   \begin{equation*}
      L(f) = \int_{X} f(x) \,d\mu(x), \quad \forall f \in C(X).
   \end{equation*}
   Moreover, the norm of $L$ is the total variation of $\mu$:
   \begin{equation}
      \norm{L} = \abs{\mu}(X)
   \end{equation}
\end{theorem}

\begin{theorem}[Lesbegue Bounded Convergence Theorem]
   Let $f_n$ be a sequence of uniformly bounded functions for all $n \in \Nat$
   that satisfy
   \begin{equation}
      \lim_{n\to\infty} f_n(x) = f(x), \quad \text{pointwise}
   \end{equation}
   Then, 
   \begin{equation}
      \lim_{n\to\infty}\int f_n \,d\mu = \int \lim_{n\to\infty} f_n \,d\mu = \int f \,d\mu
   \end{equation}
\end{theorem}

\begin{theorem}[Lusin's Theorem]
   \label{thm:lusin}
   Let $X$ be a locally compact Hausdorff space, let $\mu$ be a Radon measure on
   $X$ and $f$ is a measurable function $f: X \to \R$. Suppose that there is a
   set $A \subseteq X$ with finite measure such that $f(x)=0$ if $x\not\in A$.
   Then for every $\epsilon >0$, there exists a compactly supported continuous
   function $g: X \to \R$ with $\norm{g}_{infty} \leq \norm{f}_{\infty}$ such that
   \begin{equation}
      \mu(\{ x\in X \mid f(x) \not= g(x) \}) < \epsilon
   \end{equation}
\end{theorem}

\begin{theorem}[Discrepancy of a set]
   Let $(X, \mcal{R})$ be a set system and $\mcal{R} = P(X)$ is the power set of
   $X$. Let a mapping $\chi: X \to \{-1, +1\}$ and we name it a
   \textit{coloring} of $X$
   \begin{equation}
      \chi(A) = \sum_{x\in A} \chi(x), A \subseteq X.
   \end{equation}
   
   The \textit{discrepancy} of $\chi$ on $\mcal{R}$ is given by 
   \begin{equation}
      \disc(\mcal{R}, \chi) = \max_{R\in \mcal{R}} \abs{\chi(R)}.
   \end{equation}

   The discrepancy of $\mcal{R}$ is defined as
   \begin{align}
      \disc(\mcal{R}) 
      &= \min_{\chi: X\to\{-1,+1\}} \Big\{
         \disc(\mcal{R}, \chi)
      \Big\}
      = \min_{\chi: X\to\{-1,+1\}} \max_{R\in\mcal{R}} \abs{\chi(R)} \\
      &= \min_{\chi: X\to\{-1,+1\}} \max_{R\in\mcal{R}} 
      \abs{\sum_{x\in R} \chi(x)}.
   \end{align}
\end{theorem}

\begin{definition}[VC-class]
   VC-dimension of the set system $(X, \mcal{R})$ is defined as the maximum size
   of a shattered subset of $X$. A subset $S \subset X$ is called a
   $\epsilon$-net provided that $S \bigcap R \not= \emptyset$ for every set $R
   \subset \mcal{R}$.
\end{definition}

\begin{proposition}
   Let $X=\{1, \dots, N\}$ and a subset $R_x \subset X$ is
   \begin{equation}
      R_x = \{i: a_i \tr x + b_i \geq 0, \quad a_i \in \R^d, b_i \in \R\}.
   \end{equation}
   Let $\mcal{R} = \{R_x: x \in \R^d\}$. Then VC-dimension of the set system of
   $(X, \mcal{R})$ is at most d.
\end{proposition}

\begin{theorem}
   \cite[Theorem 1.2, p. 595]{matousekDiscrepancyApproximationsBounded1993}
   \label{thm:coloring_set_bound}
   Let $d$ be the VC-dimension of a set system $(X, \mcal{R})$. Then for every
   $N > 1$, there is a coloring of the set $X$, $\chi: X \to \{-1,1\}$ such that
   \begin{equation}
      \max_{R \in \mcal{R}} \abs{\sum_{i\in R}\chi(i)} \lesssim N^{1/2 - 1/2d}.
   \end{equation}
\end{theorem}

%%% Local Variables: 
%%% mode: latex
%%% TeX-master: "MasterThesisSfS"
%%% End: 
