\chapter{Functional Analysis And Measure Theory}
\label{app:function_measure}

We will briefly review some of the constructions and theorems of functional
analysis and measure theory used throughout this paper. Further background
material can be found.

\section{Functional Analysis}

Familiarity with the basic notions of topology is assumed (inner product
spaces, normed spaces, Banach and Hilbert spaces)

% https://www.math.uzh.ch/gorodnik/functional_analysis/lecture4.pdf


\begin{definition}[Linear opertor]
   Let $X$ and $Y$ be normed spaces over $\R$\footnote{
       It can be extended to $\mathbb{C}$
   }. $T: X \to Y$ is a linear operator if $T$ is linear, i.e.
   \begin{align*}
      T(x+y)       &= T(x) + T(y) \quad \forall x \in X, y \in Y \\
      T(\lambda x) &= \lambda T(x) \quad \forall x \in X, \lambda \in \Nat
   \end{align*}
\end{definition}

\begin{definition}[Bounded linear operator]
   Let $X$ and $Y$ be normed spaces. A linear operator $T: X \to Y$ is bounded
   if there exist a $M > 0$ such that for all $x \in X$,
   \begin{equation*}
      \norm{T(x)}_{Y} \leq M \norm{x}_{X}
   \end{equation*}
\end{definition}

A linear operator between normed spaces is bounded if and only if it is
continuous. We use $B(X, Y)$ to denote the space of Bounded operator between $X$
and $Y$.

\begin{definition}[Linear functional]
   Let $X$ be a normed space over $\R$ and $T: X \to \R$ is a linear operator,
   then $T$ is a linear functional on $X$.
\end{definition}

A linear functional $T$ is \textit{bounded} if and only if there exists a $\lambda >
0$ such that $T(x) \leq \lambda \norm{x}_{X}$ for all $x \in X$.

\begin{definition}[Dual space]
   The dual space of normed space $X$ is the vector space $X^*$ whose elements
   are the continuous linear functionals on $X$. In other words, $X^* = B(X,
   \R)$.
\end{definition}

\begin{theorem}[Hahn-Banach Theorem 1]
   \label{thm:hahn_banach_1}
   Let $(X, \norm{\cdot}_{X})$ be a subspace of $(Y, \norm{\cdot}_{Y})$ and $x_0
   \in X$. $x_0$ is in the closure $\closure{X}$ of $X$ if and only if there is
   no bounded linear functional $T: Y \to \R$ on $(Y, \norm{\cdot}_{Y})$ such
   that $T(x) = 0$ for all $x \in X$ while $F(x_0) \not= 0$
\end{theorem}

\begin{theorem}
   \label{thm:hahn_banach_2}
   Let $(X, \norm{\cdot}_{X})$ be a subspace of $(Y, \norm{\cdot}_{Y})$. Suppose
   every bounded linear functional $T: Y \to \R$, with $F(x) = 0$ for all $x \in
   X$, is identically zero on $Y$. Then $X$ is dense in $Y$.
\end{theorem}

\begin{theorem}[Riesz Representation Theorem~\cite{rudinFunctionalAnalysis1991}]
   \label{thm:riesz_rep}
   For every bounded linear functional $L$ on $C(I_n)$, there exists a unique
   signed regular Borel measure $\mu$ such that:
   \begin{equation*}
      L(f) = \int_{I_n} f(x) d\mu(x), \quad \forall f \in C(I_n).
   \end{equation*}
\end{theorem}

\section{Measure Theory}

We assume familiarity with notions such as measures, $\sigma$-algebra, \dots

\begin{definition}[Signed Borel measure]
   A 
\end{definition}

\begin{definition}[Measurable function]
   
\end{definition}

What is a bounded measurable function?

\begin{definition}[Polish space]
   
\end{definition}

%%% Local Variables: 
%%% mode: latex
%%% TeX-master: "MasterThesisSfS"
%%% End: 
